%% Generated by Sphinx.
\def\sphinxdocclass{jupyterBook}
\documentclass[letterpaper,10pt,english]{jupyterBook}
\ifdefined\pdfpxdimen
   \let\sphinxpxdimen\pdfpxdimen\else\newdimen\sphinxpxdimen
\fi \sphinxpxdimen=.75bp\relax
\ifdefined\pdfimageresolution
    \pdfimageresolution= \numexpr \dimexpr1in\relax/\sphinxpxdimen\relax
\fi
%% let collapsible pdf bookmarks panel have high depth per default
\PassOptionsToPackage{bookmarksdepth=5}{hyperref}
%% turn off hyperref patch of \index as sphinx.xdy xindy module takes care of
%% suitable \hyperpage mark-up, working around hyperref-xindy incompatibility
\PassOptionsToPackage{hyperindex=false}{hyperref}
%% memoir class requires extra handling
\makeatletter\@ifclassloaded{memoir}
{\ifdefined\memhyperindexfalse\memhyperindexfalse\fi}{}\makeatother

\PassOptionsToPackage{booktabs}{sphinx}
\PassOptionsToPackage{colorrows}{sphinx}

\PassOptionsToPackage{warn}{textcomp}

\catcode`^^^^00a0\active\protected\def^^^^00a0{\leavevmode\nobreak\ }
\usepackage{cmap}
\usepackage{fontspec}
\defaultfontfeatures[\rmfamily,\sffamily,\ttfamily]{}
\usepackage{amsmath,amssymb,amstext}
\usepackage{polyglossia}
\setmainlanguage{english}



\setmainfont{FreeSerif}[
  Extension      = .otf,
  UprightFont    = *,
  ItalicFont     = *Italic,
  BoldFont       = *Bold,
  BoldItalicFont = *BoldItalic
]
\setsansfont{FreeSans}[
  Extension      = .otf,
  UprightFont    = *,
  ItalicFont     = *Oblique,
  BoldFont       = *Bold,
  BoldItalicFont = *BoldOblique,
]
\setmonofont{FreeMono}[
  Extension      = .otf,
  UprightFont    = *,
  ItalicFont     = *Oblique,
  BoldFont       = *Bold,
  BoldItalicFont = *BoldOblique,
]



\usepackage[Bjarne]{fncychap}
\usepackage[,numfigreset=1,mathnumfig]{sphinx}

\fvset{fontsize=\small}
\usepackage{geometry}


% Include hyperref last.
\usepackage{hyperref}
% Fix anchor placement for figures with captions.
\usepackage{hypcap}% it must be loaded after hyperref.
% Set up styles of URL: it should be placed after hyperref.
\urlstyle{same}


\usepackage{sphinxmessages}


\usepackage[ backend=biber, style=apa, sorting=ynt ]{biblatex}
\addbibresource{references.bib}

        % Start of preamble defined in sphinx-jupyterbook-latex %
         \usepackage[Latin,Greek]{ucharclasses}
        \usepackage{unicode-math}
        % fixing title of the toc
        \addto\captionsenglish{\renewcommand{\contentsname}{Contents}}
        \hypersetup{
            pdfencoding=auto,
            psdextra
        }
        % End of preamble defined in sphinx-jupyterbook-latex %


\title{CMIP Greenhouse Gas (GHG) Concentration Historical Dataset}
\date{Oct 29, 2025}
\release{}
\author{Zebedee Nicholls, Florence Bockting, Mika Pflüger}
\newcommand{\sphinxlogo}{\vbox{}}
\renewcommand{\releasename}{}
\makeindex
\begin{document}

\pagestyle{empty}
\sphinxmaketitle
\pagestyle{plain}
\sphinxtableofcontents
\pagestyle{normal}
\phantomsection\label{\detokenize{intro::doc}}


\sphinxstepscope


\chapter{Overview}
\label{\detokenize{user-guide-historical:overview}}\label{\detokenize{user-guide-historical::doc}}
\sphinxAtStartPar
Here we provide a short description of the historical dataset
and a guide for users.
This is intended to provide a short introduction for users of the data:
its construction, key features, metadata
and relationship to CMIP6 forcing data.
The full details of the dataset’s construction
and evaluation against other data sources
will be provided in the full manuscript which is being prepared.


\chapter{Dataset construction}
\label{\detokenize{user-guide-historical:dataset-construction}}
The dataset is constructed following the methodology of
\cite{meinshausen_historical_2017}.

\sphinxAtStartPar
The dataset is constructed following the methodology of
\cite{meinshausen_historical_2017}.
The methods are described in full in that paper
and will be clarified and described again
in the forthcoming manuscript describing this dataset’s construction.

\sphinxAtStartPar
In brief, the dataset for each greenhouse gas is constructed via the following steps:
\begin{enumerate}
\sphinxsetlistlabels{\arabic}{enumi}{enumii}{}{.}%
\item {}
\sphinxAtStartPar
collect as many ground\sphinxhyphen{}based observations as possible

\item {}
\sphinxAtStartPar
from ground\sphinxhyphen{}based networks such as the NOAA (TODO REF)
and AGAGE (TODO REF) networks
\begin{itemize}
\item {}
\sphinxAtStartPar
these are only available over the last few decades at most
(less for some greenhouse gases)

\item {}
\sphinxAtStartPar
these are spatially sparse because sampling stations
are discrete points and there are not an infinite number of stations
(at most, usually around 30, often far fewer)

\end{itemize}

\item {}
\sphinxAtStartPar
bin the ground\sphinxhyphen{}based observations in space and time,
averaging over input stations and observations that fall in the same cell

\item {}
\sphinxAtStartPar
interpolate the binned data in space, to derive a dataset with spatial coverage

\item {}
\sphinxAtStartPar
use the interpolated, ground\sphinxhyphen{}based data
to derive a statistical model for seasonal variation and latitudinal gradients
specific to each greenhouse gas
\begin{itemize}
\item {}
\sphinxAtStartPar
the exact form of the statistical model varies by gas,
but is generally driven by either concentrations of the gas itself,
global\sphinxhyphen{}mean temperature or purely statistical regressions/extensions

\end{itemize}

\item {}
\sphinxAtStartPar
use the models, plus ice core or other proxy records,
to extend global\sphinxhyphen{}mean concentrations, seasonality and latitudinal gradients
over the full time period of the dataset (i.e. back to year 1)
\begin{itemize}
\item {}
\sphinxAtStartPar
where ice cores or proxy records are not available,
purely statistical extrapolations are used instead

\item {}
\sphinxAtStartPar
the extension varies by gas,
aiming to make use of as much information as is possible
e.g. hemisphere specific ice core information
and the latitudinal gradient
over the period covered by ground\sphinxhyphen{}based observations

\end{itemize}

\item {}
\sphinxAtStartPar
combine the extended global\sphinxhyphen{}mean, seasonality and latitudinal gradients
to create a dataset that extends over the period
year 1 to 2022 (the last year available for some observational networks
at the time the data was compiled)
\begin{itemize}
\item {}
\sphinxAtStartPar
this dataset is on our binned grid,
which we choose to be a grid comprised of latitudinal bins 15\sphinxhyphen{}degrees in size

\item {}
\sphinxAtStartPar
it is not trivial to infer the global\sphinxhyphen{}means,
seasonality and latitudinal gradient used to construct the dataset
from the output dataset. For this reason,
we include these components separately
in the \sphinxhref{https://doi.org/10.5281/zenodo.14892947}{zenodo record}
{[}TODO better ref{]}
that archives the output dataset,
all its inputs and intermediate data prdoucts

\end{itemize}

\item {}
\sphinxAtStartPar
calculate annual\sphinxhyphen{}, hemispheric\sphinxhyphen{} and global\sphinxhyphen{}means
to produce our lower resolution data products
\begin{itemize}
\item {}
\sphinxAtStartPar
we can also produce higher spatial resolution data products,
but have not done so at the moment to save processing and storage space
given that there has been no demand for these products from modelling teams

\end{itemize}

\end{enumerate}

\sphinxAtStartPar
The input datasets and associated references
are documented in the \sphinxcode{\sphinxupquote{references*}} attributes of each netCDF file.
This documentation is limited, so cannot document how each input dataset is used
(that is the role of the manuscript),
but does provide machine\sphinxhyphen{}readable provenance information
(which is used to support links between all the input data
e.g. linking of the Zenodo archive underpinning this dataset).


\chapter{Finding and accessing the data}
\label{\detokenize{user-guide-historical:finding-and-accessing-the-data}}

\section{ESGF}
\label{\detokenize{user-guide-historical:esgf}}
\sphinxAtStartPar
The \sphinxstylestrong{Earth System Grid Federation} (ESGF, REF\sphinxhyphen{}TODO) provides access to a
range of climate data.
The historical data of interest here,
which is the data to be used
for historical and piControl simulations within CMIP {[}TODO ref Dunne paper{]},
can be found under the “source ID”, \sphinxcode{\sphinxupquote{CR\sphinxhyphen{}CMIP\sphinxhyphen{}1\sphinxhyphen{}0\sphinxhyphen{}0}}.
The concept of a “source ID” is a bit of a perculiar one
to CMIP forcings data.
In practice, it is simply a unique identifier for a collection of datasets
(and it’s best not to read more than that into it).

\sphinxAtStartPar
It is possible to filter searches on ESGF
via the user interface (see ESGF user guides%
\begin{footnote}[1]\sphinxAtStartFootnote
\sphinxurl{https://esgf.github.io/esgf-user-support/user\_guide.html\#data-search-and-download}
%
\end{footnote}).
Alternatively, searches can be encoded in URLs. However, a caveat with this
approach is that URLs sometimes move, so we make no guarantee that this link
will always be live. An example provides the following link:
\begin{quote}

\sphinxAtStartPar
\DUrole{xref,myst}{https://esgf\sphinxhyphen{}node.ornl.gov/search?project=input4MIPs\&activeFacets=\%7B\%2ource\_id\%22\%3A\%22CR\sphinxhyphen{}CMIP\sphinxhyphen{}1\sphinxhyphen{}0\sphinxhyphen{}0\%22\%7D}
\end{quote}

\sphinxAtStartPar
To download the data, we recommend accessing it directly via the ESGF user interfaces
via links like the one above.
Alternately, there are tools dedicated to accessing ESGF data,
with two prominent examples being \sphinxstylestrong{esgpull}%
\begin{footnote}[2]\sphinxAtStartFootnote
\sphinxurl{https://esgf.github.io/esgf-download}
%
\end{footnote} and \sphinxstylestrong{intake\sphinxhyphen{}esgf}%
\begin{footnote}[3]\sphinxAtStartFootnote
\sphinxurl{https://intake-esgf.readthedocs.io}
%
\end{footnote}.
Please refer to the tools’ docs for usage instructions.


\section{Zenodo}
\label{\detokenize{user-guide-historical:zenodo}}
\sphinxAtStartPar
While it aims to be, the ESGF is technically not a permanent archive
and does not issue DOIs.
In order to provide more reliable, citable access to the data,
we also provide it on \sphinxstylestrong{Zenodo} (REF\sphinxhyphen{}TODO).
The data, as well as all the source code and input data used to process it,
can be found at \sphinxurl{https://doi.org/10.5281/zenodo.14892947}.


\chapter{Data description}
\label{\detokenize{user-guide-historical:data-description}}

\section{Format}
\label{\detokenize{user-guide-historical:format}}
\sphinxAtStartPar
The data is provided in \sphinxstylestrong{netCDF format} {[}TODO citation{]}.
This self\sphinxhyphen{}describing format allows the data
to be placed in the same file as metadata
(in the so\sphinxhyphen{}called “file header”).
To facilitate simpler use of the data,
each dataset is split across multiple files.
The advantage of this is that users do not need to load all years of data
if they are only interested in data for a certain range,
which can significantly improve data loading times.
To get the complete dataset,
the files can simply be concatenated in time.


\section{Grids and frequencies provided}
\label{\detokenize{user-guide-historical:grids-and-frequencies-provided}}
\sphinxAtStartPar
We provide five combinations of grids and time sampling
(also referred to as frequency,
although this is a bit of a misuse as the units of frequency are per time,
which doesn’t match the convention for these metadata values).
The grid and frequency information for each file can be found in its netCDF header
under the attributes \sphinxcode{\sphinxupquote{grid\_label}} (for grid) and \sphinxcode{\sphinxupquote{frequency}} (for time sampling).
The \sphinxcode{\sphinxupquote{grid\_label}} and \sphinxcode{\sphinxupquote{frequency}} also appear in each file’s name,
which allows files to be filtered without needing to load them first.

\sphinxAtStartPar
The five combinations of grid and time sampling are:
\begin{enumerate}
\sphinxsetlistlabels{\arabic}{enumi}{enumii}{}{.}%
\item {}
\sphinxAtStartPar
global\sphinxhyphen{}, annual\sphinxhyphen{}mean (\sphinxcode{\sphinxupquote{grid\_label="gm"}}, \sphinxcode{\sphinxupquote{frequency="yr"}})

\item {}
\sphinxAtStartPar
global\sphinxhyphen{}, monthly\sphinxhyphen{}mean (\sphinxcode{\sphinxupquote{grid\_label="gm"}}, \sphinxcode{\sphinxupquote{frequency="mon"}})

\item {}
\sphinxAtStartPar
hemispheric\sphinxhyphen{}, annual\sphinxhyphen{}mean (\sphinxcode{\sphinxupquote{grid\_label="gr1z"}}, \sphinxcode{\sphinxupquote{frequency="yr"}})

\item {}
\sphinxAtStartPar
hemispheric\sphinxhyphen{}, monthly\sphinxhyphen{}mean (\sphinxcode{\sphinxupquote{grid\_label="gr1z"}}, \sphinxcode{\sphinxupquote{frequency="mon"}})

\item {}
\sphinxAtStartPar
15\sphinxhyphen{}degree latitudinal, monthly\sphinxhyphen{}mean (\sphinxcode{\sphinxupquote{grid\_label="gnz"}}, \sphinxcode{\sphinxupquote{frequency="mon"}})

\end{enumerate}


\section{Species provided}
\label{\detokenize{user-guide-historical:species-provided}}
\sphinxAtStartPar
We provide concentrations for 43 greenhouse gas concentrations and species,
as well as three equivalent species.
The species are:


\begin{itemize}
\item {}
\sphinxAtStartPar
major greenhouse gases (3)
\begin{itemize}
\item {}
\sphinxAtStartPar
CH4, CO2, N2O

\end{itemize}

\item {}
\sphinxAtStartPar
ozone\sphinxhyphen{}depleting substances (17)
\begin{itemize}
\item {}
\sphinxAtStartPar
CFCs (5)
\begin{itemize}
\item {}
\sphinxAtStartPar
CFC\sphinxhyphen{}11, CFC\sphinxhyphen{}113, CFC\sphinxhyphen{}114, CFC\sphinxhyphen{}115, CFC\sphinxhyphen{}12

\end{itemize}

\item {}
\sphinxAtStartPar
HCFCs (3)
\begin{itemize}
\item {}
\sphinxAtStartPar
HCFC\sphinxhyphen{}141b, HCFC\sphinxhyphen{}142b, HCFC\sphinxhyphen{}22

\end{itemize}

\item {}
\sphinxAtStartPar
Halons (3)
\begin{itemize}
\item {}
\sphinxAtStartPar
Halon 1211, Halon 1301, Halon 2402

\end{itemize}

\item {}
\sphinxAtStartPar
other ozone\sphinxhyphen{}depleting substances (6)
\begin{itemize}
\item {}
\sphinxAtStartPar
CCl4, CH2Cl2, CH3Br,
CH3CCl3, CH3Cl, CHCl3

\end{itemize}

\end{itemize}

\item {}
\sphinxAtStartPar
ozone fluorinated compounds (23)
\begin{itemize}
\item {}
\sphinxAtStartPar
HFCs (11)
\begin{itemize}
\item {}
\sphinxAtStartPar
HFC\sphinxhyphen{}125, HFC\sphinxhyphen{}134a, HFC\sphinxhyphen{}143a, HFC\sphinxhyphen{}152a, HFC\sphinxhyphen{}227ea, HFC\sphinxhyphen{}23, HFC\sphinxhyphen{}236fa,
HFC\sphinxhyphen{}245fa, HFC\sphinxhyphen{}32, HFC\sphinxhyphen{}365mfc, HFC\sphinxhyphen{}4310mee

\end{itemize}

\item {}
\sphinxAtStartPar
PFCs (9)
\begin{itemize}
\item {}
\sphinxAtStartPar
C2F6, C3F8,
C4F10, C5F12,
C6F14, C7F16,
C8F18, CC4F8,
CF4

\end{itemize}

\item {}
\sphinxAtStartPar
other (3)
\begin{itemize}
\item {}
\sphinxAtStartPar
NF3, SF6, SO2F2

\end{itemize}

\end{itemize}

\end{itemize}


\subsection{Equivalent species}
\label{\detokenize{user-guide-historical:equivalent-species}}
\sphinxAtStartPar
For most models, you will not use all 43 species.
As a result, we provide equivalent species too.
There are two options if you don’t want to use all 43 species.


\subsubsection{Option 1}
\label{\detokenize{user-guide-historical:option-1}}
\sphinxAtStartPar
Use CO2, CH4, N2O and CFC\sphinxhyphen{}12 directly.
Use CFC\sphinxhyphen{}11 equivalent to capture the radiative effect of all other species.


\subsubsection{Option 2}
\label{\detokenize{user-guide-historical:option-2}}
\sphinxAtStartPar
Use CO2, CH4 and N2O directly.
Use CFC\sphinxhyphen{}12 equivalent
to capture the radiative effect of all ozone depleting substances (ODSs)
and HFC\sphinxhyphen{}134a equivalent
to capture the radiative effect of all other fluorinated gases.


\section{Uncertainty}
\label{\detokenize{user-guide-historical:uncertainty}}
\sphinxAtStartPar
At present, we provide no analysis of the uncertainty associated with these datasets.
In radiative forcing terms, the uncertainty in these concentrations
is very likely to be small compared to other uncertainties in the climate system,
but this statement is not based on any robust analysis
(rather it is based on expert judgement).
It is also worth noting that the uncertainty increases as we go further back in time,
particularly as we shift from using surface flasks to relying on ice cores instead.


\section{Differences compared to CMIP6}
\label{\detokenize{user-guide-historical:differences-compared-to-cmip6}}
\sphinxAtStartPar
At present, the changes from CMIP6 are minor,
with the maximum difference in effective radiative forcing terms
being 0.05 W / m2
(and generally much smaller than this, particularly after 1850).
For more details, see the plots in the user guide below
and the forthcoming manuscript.


\chapter{User guide}
\label{\detokenize{user-guide-historical:user-guide}}
\sphinxAtStartPar
Having downloaded the data, using it is quite straightforward.


\section{Annual\sphinxhyphen{}, global\sphinxhyphen{}mean data}
\label{\detokenize{user-guide-historical:annual-global-mean-data}}
\sphinxAtStartPar
We start with the annual\sphinxhyphen{}, global\sphinxhyphen{}mean data.
Like all our datasets, this is composed of three files,
each covering a different time period:
\begin{enumerate}
\sphinxsetlistlabels{\arabic}{enumi}{enumii}{}{.}%
\item {}
\sphinxAtStartPar
year 1 to year 999

\item {}
\sphinxAtStartPar
year 1000 to year 1749

\item {}
\sphinxAtStartPar
year 1750 to year 2022

\end{enumerate}

\sphinxAtStartPar
For yearly data, the time labels in the filename are years
(for months, the month is included e.g. you will see \sphinxcode{\sphinxupquote{000101\sphinxhyphen{}09912}}
rather than \sphinxcode{\sphinxupquote{0001\sphinxhyphen{}0999}} in the filename,
the files also have different values for the \sphinxcode{\sphinxupquote{frequency}} attribute).
Global\sphinxhyphen{}mean data is identified by the ‘grid label’ \sphinxcode{\sphinxupquote{gm}},
which appears in the filename.
Below we show the filenames for the CO2 output.

\begin{sphinxuseclass}{cell}
\begin{sphinxuseclass}{tag_remove_input}\begin{sphinxVerbatimOutput}

\begin{sphinxuseclass}{cell_output}
\begin{sphinxVerbatim}[commandchars=\\\{\}]
\PYGZhy{} co2\PYGZus{}input4MIPs\PYGZus{}GHGConcentrations\PYGZus{}CMIP\PYGZus{}CR\PYGZhy{}CMIP\PYGZhy{}1\PYGZhy{}0\PYGZhy{}0\PYGZus{}gm\PYGZus{}1000\PYGZhy{}1749.nc
\PYGZhy{} co2\PYGZus{}input4MIPs\PYGZus{}GHGConcentrations\PYGZus{}CMIP\PYGZus{}CR\PYGZhy{}CMIP\PYGZhy{}1\PYGZhy{}0\PYGZhy{}0\PYGZus{}gm\PYGZus{}1750\PYGZhy{}2022.nc
\PYGZhy{} co2\PYGZus{}input4MIPs\PYGZus{}GHGConcentrations\PYGZus{}CMIP\PYGZus{}CR\PYGZhy{}CMIP\PYGZhy{}1\PYGZhy{}0\PYGZhy{}0\PYGZus{}gm\PYGZus{}0001\PYGZhy{}0999.nc
\end{sphinxVerbatim}

\end{sphinxuseclass}\end{sphinxVerbatimOutput}

\end{sphinxuseclass}
\end{sphinxuseclass}
\sphinxAtStartPar
Output for other gases are named identically,
with \sphinxcode{\sphinxupquote{co2}} being replaced by the other gas name.
For example, for methane the filenames are:

\begin{sphinxuseclass}{cell}
\begin{sphinxuseclass}{tag_remove_input}\begin{sphinxVerbatimOutput}

\begin{sphinxuseclass}{cell_output}
\begin{sphinxVerbatim}[commandchars=\\\{\}]
\PYGZhy{} ch4\PYGZus{}input4MIPs\PYGZus{}GHGConcentrations\PYGZus{}CMIP\PYGZus{}CR\PYGZhy{}CMIP\PYGZhy{}1\PYGZhy{}0\PYGZhy{}0\PYGZus{}gm\PYGZus{}1000\PYGZhy{}1749.nc
\PYGZhy{} ch4\PYGZus{}input4MIPs\PYGZus{}GHGConcentrations\PYGZus{}CMIP\PYGZus{}CR\PYGZhy{}CMIP\PYGZhy{}1\PYGZhy{}0\PYGZhy{}0\PYGZus{}gm\PYGZus{}1750\PYGZhy{}2022.nc
\PYGZhy{} ch4\PYGZus{}input4MIPs\PYGZus{}GHGConcentrations\PYGZus{}CMIP\PYGZus{}CR\PYGZhy{}CMIP\PYGZhy{}1\PYGZhy{}0\PYGZhy{}0\PYGZus{}gm\PYGZus{}0001\PYGZhy{}0999.nc
\end{sphinxVerbatim}

\end{sphinxuseclass}\end{sphinxVerbatimOutput}

\end{sphinxuseclass}
\end{sphinxuseclass}
\sphinxAtStartPar
As described above, the data is netCDF files.
This means that metadata can be trivially inspected
using a tool like \sphinxcode{\sphinxupquote{ncdump}}.
As you can see, there is a lot of metadata included in these files.
In general, you should not need to parse this metadata directly.
However, if you have specific questions,
please feel free to contact the emails given in the \sphinxcode{\sphinxupquote{contact}} attribute.

\begin{sphinxuseclass}{cell}
\begin{sphinxuseclass}{tag_remove_input}\begin{sphinxVerbatimOutput}

\begin{sphinxuseclass}{cell_output}
\begin{sphinxVerbatim}[commandchars=\\\{\}]
netcdf co2\PYGZus{}input4MIPs\PYGZus{}GHGConcentrations\PYGZus{}CMIP\PYGZus{}CR\PYGZhy{}CMIP\PYGZhy{}1\PYGZhy{}0\PYGZhy{}0\PYGZus{}gm\PYGZus{}1000\PYGZhy{}1749 \PYGZob{}
dimensions:
	time = UNLIMITED ; // (750 currently)
	bnds = 2 ;
variables:
	float co2(time) ;
		co2:standard\PYGZus{}name = \PYGZdq{}mole\PYGZus{}fraction\PYGZus{}of\PYGZus{}carbon\PYGZus{}dioxide\PYGZus{}in\PYGZus{}air\PYGZdq{} ;
		co2:long\PYGZus{}name = \PYGZdq{}co2\PYGZdq{} ;
		co2:units = \PYGZdq{}ppm\PYGZdq{} ;
		co2:cell\PYGZus{}methods = \PYGZdq{}area: time: mean\PYGZdq{} ;
	double time(time) ;
		time:axis = \PYGZdq{}T\PYGZdq{} ;
		time:bounds = \PYGZdq{}time\PYGZus{}bnds\PYGZdq{} ;
		time:units = \PYGZdq{}days since 1850\PYGZhy{}01\PYGZhy{}01\PYGZdq{} ;
		time:standard\PYGZus{}name = \PYGZdq{}time\PYGZdq{} ;
		time:calendar = \PYGZdq{}proleptic\PYGZus{}gregorian\PYGZdq{} ;
		time:amip = \PYGZdq{}time\PYGZdq{} ;
	double time\PYGZus{}bnds(time, bnds) ;

// global attributes:
		:Conventions = \PYGZdq{}CF\PYGZhy{}1.7\PYGZdq{} ;
		:activity\PYGZus{}id = \PYGZdq{}input4MIPs\PYGZdq{} ;
		:comment = \PYGZdq{}Data compiled by Climate Resource, based on science
by many others (see \PYGZbs{}\PYGZsq{}references*\PYGZbs{}\PYGZsq{} attributes). For funding information, see
the \PYGZbs{}\PYGZsq{}funding*\PYGZbs{}\PYGZsq{} attributes.\PYGZdq{} ;
		:contact =
\PYGZdq{}zebedee.nicholls@climate\PYGZhy{}resource.com;malte.meinshausen@climate\PYGZhy{}resource.com\PYGZdq{} ;
		:creation\PYGZus{}date = \PYGZdq{}2025\PYGZhy{}02\PYGZhy{}28T18:23:33Z\PYGZdq{} ;
		:dataset\PYGZus{}category = \PYGZdq{}GHGConcentrations\PYGZdq{} ;
		:doi = \PYGZdq{}https://doi.org/10.5281/zenodo.14892947\PYGZdq{} ;
		:frequency = \PYGZdq{}yr\PYGZdq{} ;
		:funding = \PYGZdq{}Financial support has been provided by the CMIP
International Project Office (CMIP IPO), which is hosted by the European Space
Agency (ESA), with staff provided by HE Space Operations Ltd. This research has
been funded by the European Space Agency (ESA) as part of the GHG Forcing For
CMIP project of the Climate Change Initiative (CCI) (ESA Contract No.
4000146681/24/I\PYGZhy{}LR\PYGZhy{}cl).\PYGZdq{} ;
		:funding\PYGZus{}short\PYGZus{}names = \PYGZdq{}Quick DECK GHG Forcing \PYGZhy{}\PYGZhy{}\PYGZhy{} GHG Forcing
For CMIP\PYGZdq{} ;
		:funding\PYGZus{}urls = \PYGZdq{}No URL \PYGZhy{}\PYGZhy{}\PYGZhy{}
climate.esa.int/supporting\PYGZhy{}modelling/cmip\PYGZhy{}forcing\PYGZhy{}ghg\PYGZhy{}concentrations/\PYGZdq{} ;
		:further\PYGZus{}info\PYGZus{}url =
\PYGZdq{}https://github.com/climate\PYGZhy{}resource/CMIP\PYGZhy{}GHG\PYGZhy{}Concentration\PYGZhy{}Generation\PYGZdq{} ;
		:grid\PYGZus{}label = \PYGZdq{}gm\PYGZdq{} ;
		:institution\PYGZus{}id = \PYGZdq{}CR\PYGZdq{} ;
		:license = \PYGZdq{}The input4MIPs data linked to this entry is
licensed under a Creative Commons Attribution 4.0 International
(https://creativecommons.org/licenses/by/4.0/). Consult
https://pcmdi.llnl.gov/CMIP6/TermsOfUse for terms of use governing CMIP6Plus
output, including citation requirements and proper acknowledgment. The data
producers and data providers make no warranty, either express or implied,
including, but not limited to, warranties of merchantability and fitness for a
particular purpose. All liabilities arising from the supply of the information
(including any liability arising in negligence) are excluded to the fullest
extent permitted by law.\PYGZdq{} ;
		:license\PYGZus{}id = \PYGZdq{}CC BY 4.0\PYGZdq{} ;
		:mip\PYGZus{}era = \PYGZdq{}CMIP7\PYGZdq{} ;
		:nominal\PYGZus{}resolution = \PYGZdq{}10000 km\PYGZdq{} ;
		:product = \PYGZdq{}derived\PYGZdq{} ;
		:realm = \PYGZdq{}atmos\PYGZdq{} ;
		:references = \PYGZdq{}C. D. Keeling, S. C. Piper, ..., M. Heimann, and
H. A. Meijer, Exchanges of atmospheric CO2 and 13CO2 with the terrestrial
biosphere and  oceans from 1978 to 2000. I. Global aspects, SIO Reference
Series, No. 01\PYGZhy{}06, Scripps Institution of Oceanography, San Diego, 88 pages,
2001. \PYGZhy{}\PYGZhy{}\PYGZhy{} Nicholls, Z., Meinshausen, M., Lewis, J., Pflueger, M., Menking, A.,
...: Greenhouse gas concentrations for climate modelling (CMIP7), in\PYGZhy{}prep,
2025. \PYGZhy{}\PYGZhy{}\PYGZhy{} Lan, X., J.W. Mund, A.M. Crotwell, K.W. Thoning, E. Moglia, M.
Madronich, K. Baugh, G. Petron, M.J. Crotwell, D. Neff, S. Wolter, T. Mefford
and S. DeVogel (2024), Atmospheric Carbon Dioxide Dry Air Mole Fractions from
the NOAA GML Carbon Cycle Cooperative Global Air Sampling Network, 1968\PYGZhy{}2023,
Version: 2024\PYGZhy{}07\PYGZhy{}30, https://doi.org/10.15138/wkgj\PYGZhy{}f215 \PYGZhy{}\PYGZhy{}\PYGZhy{} K.W. Thoning, A.M.
Crotwell, and J.W. Mund (2024), Atmospheric Carbon Dioxide Dry Air Mole
Fractions from continuous measurements at Mauna Loa, Hawaii, Barrow, Alaska,
American Samoa and South Pole, 1973\PYGZhy{}present. Version 2024\PYGZhy{}08\PYGZhy{}15, National
Oceanic and Atmospheric Administration (NOAA), Global Monitoring Laboratory
(GML), Boulder, Colorado, USA https://doi.org/10.15138/yaf1\PYGZhy{}bk21 \PYGZhy{}\PYGZhy{}\PYGZhy{} Menking,
J. A., Etheridge, D., ..., Spencer, D., and Caldow, C. (in prep.). Filling gaps
and reducing uncertainty in existing Law Dome ice core records. \PYGZhy{}\PYGZhy{}\PYGZhy{}
Meinshausen, M., Vogel, E., ..., Wang, R. H. J., and Weiss, R.: Historical
greenhouse gas concentrations for climate modelling (CMIP6), Geosci. Model
Dev., 10, 2057\PYGZhy{}2116, https://doi.org/10.5194/gmd\PYGZhy{}10\PYGZhy{}2057\PYGZhy{}2017, 2017. \PYGZhy{}\PYGZhy{}\PYGZhy{}
Rubino, Mauro; Etheridge, David; ... Van Ommen, Tas; \PYGZam{} Smith, Andrew (2019):
Law Dome Ice Core 2000\PYGZhy{}Year CO2, CH4, N2O and d13C\PYGZhy{}CO2. v3. CSIRO. Data
Collection. https://doi.org/10.25919/5bfe29ff807fb \PYGZhy{}\PYGZhy{}\PYGZhy{} King, A.C.F., Bauska,
T.K., Brook, E.J. et al. Reconciling ice core CO2 and land\PYGZhy{}use change following
New World\PYGZhy{}Old World contact. Nat Commun 15, 1735 (2024).
https://doi.org/10.1038/s41467\PYGZhy{}024\PYGZhy{}45894\PYGZhy{}9 \PYGZhy{}\PYGZhy{}\PYGZhy{} Morice, C. P., Kennedy, J. J.,
Rayner, N. A., Winn, J. P., Hogan, E., ..., et al. (2021). An updated
assessment of near\PYGZhy{}surface temperature change from 1850: the HadCRUT5 data set.
Journal of Geophysical Research: Atmospheres, 126, e2019JD032361.
https://doi.org/10.1029/2019JD032361 \PYGZhy{}\PYGZhy{}\PYGZhy{} Bauska, T., Joos, F., Mix, A. et al.
Links between atmospheric carbon dioxide, the land carbon reservoir and climate
over the past millennium. Nature Geosci 8, 383\PYGZhy{}387 (2015).
https://doi.org/10.1038/ngeo2422 \PYGZhy{}\PYGZhy{}\PYGZhy{} Ahn, J., E. J. Brook, A. Schmittner, and
K. Kreutz (2012), Abrupt change in atmospheric CO2 during the last ice age,
Geophys. Res. Lett., 39, L18711, doi:10.1029/2012GL053018.\PYGZdq{} ;
		:references\PYGZus{}dois = \PYGZdq{}No DOI \PYGZhy{}\PYGZhy{}\PYGZhy{} No DOI \PYGZhy{}\PYGZhy{}\PYGZhy{}
https://doi.org/10.15138/wkgj\PYGZhy{}f215 \PYGZhy{}\PYGZhy{}\PYGZhy{} https://doi.org/10.15138/yaf1\PYGZhy{}bk21 \PYGZhy{}\PYGZhy{}\PYGZhy{}
No DOI \PYGZhy{}\PYGZhy{}\PYGZhy{} https://doi.org/10.5194/gmd\PYGZhy{}10\PYGZhy{}2057\PYGZhy{}2017 \PYGZhy{}\PYGZhy{}\PYGZhy{}
https://doi.org/10.25919/5bfe29ff807fb \PYGZhy{}\PYGZhy{}\PYGZhy{}
https://doi.org/10.1038/s41467\PYGZhy{}024\PYGZhy{}45894\PYGZhy{}9 \PYGZhy{}\PYGZhy{}\PYGZhy{}
https://doi.org/10.1029/2019JD032361 \PYGZhy{}\PYGZhy{}\PYGZhy{} https://doi.org/10.1038/ngeo2422 \PYGZhy{}\PYGZhy{}\PYGZhy{}
https://doi.org/10.1029/2012GL053018\PYGZdq{} ;
		:references\PYGZus{}short\PYGZus{}names = \PYGZdq{}Scripps \PYGZhy{} Law Dome merged CO2 record
\PYGZhy{}\PYGZhy{}\PYGZhy{} Nicholls et al., 2025 (in\PYGZhy{}prep) \PYGZhy{}\PYGZhy{}\PYGZhy{} NOAA co2 surface\PYGZhy{}flask \PYGZhy{}\PYGZhy{}\PYGZhy{} NOAA co2
in\PYGZhy{}situ \PYGZhy{}\PYGZhy{}\PYGZhy{} Menking et al., 2025 (in\PYGZhy{}prep.) \PYGZhy{}\PYGZhy{}\PYGZhy{} Meinshausen et al., 2017 \PYGZhy{}\PYGZhy{}\PYGZhy{}
Law Dome ice core \PYGZhy{}\PYGZhy{}\PYGZhy{} King et al., 2024 \PYGZhy{}\PYGZhy{}\PYGZhy{} HadCRUT5 \PYGZhy{}\PYGZhy{}\PYGZhy{} Bauska et al., 2015
\PYGZhy{}\PYGZhy{}\PYGZhy{} Ahn et al., 2012\PYGZdq{} ;
		:references\PYGZus{}urls =
\PYGZdq{}https://scrippsco2.ucsd.edu/data/atmospheric\PYGZus{}co2/icecore\PYGZus{}merged\PYGZus{}products.html
\PYGZhy{}\PYGZhy{}\PYGZhy{} https://github.com/climate\PYGZhy{}resource/CMIP\PYGZhy{}GHG\PYGZhy{}Concentration\PYGZhy{}Generation \PYGZhy{}\PYGZhy{}\PYGZhy{}
https://gml.noaa.gov/aftp/data/trace\PYGZus{}gases/co2/flask/surface/co2\PYGZus{}surface\PYGZhy{}flask\PYGZus{}c
cgg\PYGZus{}text.zip \PYGZhy{}\PYGZhy{}\PYGZhy{}
https://gml.noaa.gov/aftp/data/greenhouse\PYGZus{}gases/co2/in\PYGZhy{}situ/surface/co2\PYGZus{}surface\PYGZhy{}
insitu\PYGZus{}ccgg\PYGZus{}text.zip \PYGZhy{}\PYGZhy{}\PYGZhy{} author\PYGZhy{}supplied.invalid \PYGZhy{}\PYGZhy{}\PYGZhy{}
https://doi.org/10.5194/gmd\PYGZhy{}10\PYGZhy{}2057\PYGZhy{}2017 \PYGZhy{}\PYGZhy{}\PYGZhy{}
https://doi.org/10.25919/5bfe29ff807fb \PYGZhy{}\PYGZhy{}\PYGZhy{}
https://doi.org/10.1038/s41467\PYGZhy{}024\PYGZhy{}45894\PYGZhy{}9 \PYGZhy{}\PYGZhy{}\PYGZhy{}
https://www.metoffice.gov.uk/hadobs/hadcrut5/data/HadCRUT.5.0.2.0/download.html
\PYGZhy{}\PYGZhy{}\PYGZhy{} https://doi.org/10.1038/ngeo2422 \PYGZhy{}\PYGZhy{}\PYGZhy{} https://doi.org/10.1029/2012GL053018\PYGZdq{} ;
		:source\PYGZus{}id = \PYGZdq{}CR\PYGZhy{}CMIP\PYGZhy{}1\PYGZhy{}0\PYGZhy{}0\PYGZdq{} ;
		:source\PYGZus{}version = \PYGZdq{}1.0.0\PYGZdq{} ;
		:target\PYGZus{}mip = \PYGZdq{}CMIP\PYGZdq{} ;
		:tracking\PYGZus{}id =
\PYGZdq{}hdl:21.14100/0184fccb\PYGZhy{}d590\PYGZhy{}4d18\PYGZhy{}be31\PYGZhy{}0de4b7814892\PYGZdq{} ;
		:variable\PYGZus{}id = \PYGZdq{}co2\PYGZdq{} ;
\PYGZcb{}
\end{sphinxVerbatim}

\end{sphinxuseclass}\end{sphinxVerbatimOutput}

\end{sphinxuseclass}
\end{sphinxuseclass}
\sphinxAtStartPar
Using a tool like \sphinxhref{https://github.com/pydata/xarray}{xarray},
loading and working with the data is trivial.

\begin{sphinxuseclass}{cell}\begin{sphinxVerbatimInput}

\begin{sphinxuseclass}{cell_input}
\begin{sphinxVerbatim}[commandchars=\\\{\}]
\PYG{k+kn}{import}\PYG{+w}{ }\PYG{n+nn}{xarray}\PYG{+w}{ }\PYG{k}{as}\PYG{+w}{ }\PYG{n+nn}{xr}

\PYG{n}{time\PYGZus{}coder} \PYG{o}{=} \PYG{n}{xr}\PYG{o}{.}\PYG{n}{coders}\PYG{o}{.}\PYG{n}{CFDatetimeCoder}\PYG{p}{(}\PYG{n}{use\PYGZus{}cftime}\PYG{o}{=}\PYG{k+kc}{True}\PYG{p}{)}
\PYG{n}{ds\PYGZus{}co2\PYGZus{}yearly\PYGZus{}global} \PYG{o}{=} \PYG{n}{xr}\PYG{o}{.}\PYG{n}{open\PYGZus{}mfdataset}\PYG{p}{(}\PYG{n}{co2\PYGZus{}yearly\PYGZus{}global\PYGZus{}fps}\PYG{p}{,} \PYG{n}{decode\PYGZus{}times}\PYG{o}{=}\PYG{n}{time\PYGZus{}coder}\PYG{p}{)}
\end{sphinxVerbatim}

\end{sphinxuseclass}\end{sphinxVerbatimInput}

\end{sphinxuseclass}
\begin{sphinxuseclass}{cell}\begin{sphinxVerbatimInput}

\begin{sphinxuseclass}{cell_input}
\begin{sphinxVerbatim}[commandchars=\\\{\}]
\PYG{n}{ds\PYGZus{}co2\PYGZus{}yearly\PYGZus{}global}
\end{sphinxVerbatim}

\end{sphinxuseclass}\end{sphinxVerbatimInput}
\begin{sphinxVerbatimOutput}

\begin{sphinxuseclass}{cell_output}
\begin{sphinxVerbatim}[commandchars=\\\{\}]
\PYGZlt{}xarray.Dataset\PYGZgt{} Size: 57kB
Dimensions:    (time: 2022, bnds: 2)
Coordinates:
  * time       (time) object 16kB 0001\PYGZhy{}07\PYGZhy{}02 12:00:00 ... 2022\PYGZhy{}07\PYGZhy{}02 12:00:00
Dimensions without coordinates: bnds
Data variables:
    co2        (time) float32 8kB 280.5 280.4 280.4 280.3 ... 412.9 415.1 417.3
    time\PYGZus{}bnds  (time, bnds) object 32kB 0001\PYGZhy{}01\PYGZhy{}01 00:00:00 ... 2023\PYGZhy{}01\PYGZhy{}01 00...
Attributes: (12/29)
    Conventions:             CF\PYGZhy{}1.7
    activity\PYGZus{}id:             input4MIPs
    comment:                 Data compiled by Climate Resource, based on scie...
    contact:                 zebedee.nicholls@climate\PYGZhy{}resource.com;malte.mein...
    creation\PYGZus{}date:           2025\PYGZhy{}02\PYGZhy{}28T18:23:33Z
    dataset\PYGZus{}category:        GHGConcentrations
    ...                      ...
    references\PYGZus{}urls:         https://scrippsco2.ucsd.edu/data/atmospheric\PYGZus{}co2...
    source\PYGZus{}id:               CR\PYGZhy{}CMIP\PYGZhy{}1\PYGZhy{}0\PYGZhy{}0
    source\PYGZus{}version:          1.0.0
    target\PYGZus{}mip:              CMIP
    tracking\PYGZus{}id:             hdl:21.14100/d3e2e715\PYGZhy{}f5a7\PYGZhy{}4d0e\PYGZhy{}973c\PYGZhy{}1fb77263f9b7
    variable\PYGZus{}id:             co2
\end{sphinxVerbatim}

\end{sphinxuseclass}\end{sphinxVerbatimOutput}

\end{sphinxuseclass}
\begin{sphinxuseclass}{cell}
\begin{sphinxuseclass}{tag_remove_input}\begin{sphinxVerbatimOutput}

\begin{sphinxuseclass}{cell_output}
\noindent\sphinxincludegraphics{{355eb56913fa9c80faafbbb62441b56f754a8e1747bf7bc97ae01675454d63f2}.png}

\end{sphinxuseclass}\end{sphinxVerbatimOutput}

\end{sphinxuseclass}
\end{sphinxuseclass}

\section{Space\sphinxhyphen{} and time\sphinxhyphen{}average nature of the data}
\label{\detokenize{user-guide-historical:space-and-time-average-nature-of-the-data}}
\sphinxAtStartPar
All of our data represents the mean over each cell.
This is indicated by the \sphinxcode{\sphinxupquote{cell\_methods}} attribute
of all of our output variables.

\begin{sphinxuseclass}{cell}\begin{sphinxVerbatimInput}

\begin{sphinxuseclass}{cell_input}
\begin{sphinxVerbatim}[commandchars=\\\{\}]
\PYG{n}{ds\PYGZus{}co2\PYGZus{}yearly\PYGZus{}global}\PYG{p}{[}\PYG{l+s+s2}{\PYGZdq{}}\PYG{l+s+s2}{co2}\PYG{l+s+s2}{\PYGZdq{}}\PYG{p}{]}\PYG{o}{.}\PYG{n}{attrs}\PYG{p}{[}\PYG{l+s+s2}{\PYGZdq{}}\PYG{l+s+s2}{cell\PYGZus{}methods}\PYG{l+s+s2}{\PYGZdq{}}\PYG{p}{]}
\end{sphinxVerbatim}

\end{sphinxuseclass}\end{sphinxVerbatimInput}
\begin{sphinxVerbatimOutput}

\begin{sphinxuseclass}{cell_output}
\begin{sphinxVerbatim}[commandchars=\\\{\}]
\PYGZsq{}area: time: mean\PYGZsq{}
\end{sphinxVerbatim}

\end{sphinxuseclass}\end{sphinxVerbatimOutput}

\end{sphinxuseclass}
\sphinxAtStartPar
This mean is both in space and time.
The time bounds covered by each step
are specified by the \sphinxcode{\sphinxupquote{time\_bnds}} variable
(when there is spatial information,
equivalent \sphinxcode{\sphinxupquote{lat\_bnds}} and \sphinxcode{\sphinxupquote{lon\_bnds}}
information is also included).
This variable specifies the start (inclusive)
and end (exclusive) of the time period
covered by each data point.

\begin{sphinxuseclass}{cell}\begin{sphinxVerbatimInput}

\begin{sphinxuseclass}{cell_input}
\begin{sphinxVerbatim}[commandchars=\\\{\}]
\PYG{n}{ds\PYGZus{}co2\PYGZus{}yearly\PYGZus{}global}\PYG{p}{[}\PYG{l+s+s2}{\PYGZdq{}}\PYG{l+s+s2}{time\PYGZus{}bnds}\PYG{l+s+s2}{\PYGZdq{}}\PYG{p}{]}
\end{sphinxVerbatim}

\end{sphinxuseclass}\end{sphinxVerbatimInput}
\begin{sphinxVerbatimOutput}

\begin{sphinxuseclass}{cell_output}
\begin{sphinxVerbatim}[commandchars=\\\{\}]
\PYGZlt{}xarray.DataArray \PYGZsq{}time\PYGZus{}bnds\PYGZsq{} (time: 2022, bnds: 2)\PYGZgt{} Size: 32kB
array([[cftime.DatetimeProlepticGregorian(1, 1, 1, 0, 0, 0, 0, has\PYGZus{}year\PYGZus{}zero=True),
        cftime.DatetimeProlepticGregorian(2, 1, 1, 0, 0, 0, 0, has\PYGZus{}year\PYGZus{}zero=True)],
       [cftime.DatetimeProlepticGregorian(2, 1, 1, 0, 0, 0, 0, has\PYGZus{}year\PYGZus{}zero=True),
        cftime.DatetimeProlepticGregorian(3, 1, 1, 0, 0, 0, 0, has\PYGZus{}year\PYGZus{}zero=True)],
       [cftime.DatetimeProlepticGregorian(3, 1, 1, 0, 0, 0, 0, has\PYGZus{}year\PYGZus{}zero=True),
        cftime.DatetimeProlepticGregorian(4, 1, 1, 0, 0, 0, 0, has\PYGZus{}year\PYGZus{}zero=True)],
       ...,
       [cftime.DatetimeProlepticGregorian(2020, 1, 1, 0, 0, 0, 0, has\PYGZus{}year\PYGZus{}zero=True),
        cftime.DatetimeProlepticGregorian(2021, 1, 1, 0, 0, 0, 0, has\PYGZus{}year\PYGZus{}zero=True)],
       [cftime.DatetimeProlepticGregorian(2021, 1, 1, 0, 0, 0, 0, has\PYGZus{}year\PYGZus{}zero=True),
        cftime.DatetimeProlepticGregorian(2022, 1, 1, 0, 0, 0, 0, has\PYGZus{}year\PYGZus{}zero=True)],
       [cftime.DatetimeProlepticGregorian(2022, 1, 1, 0, 0, 0, 0, has\PYGZus{}year\PYGZus{}zero=True),
        cftime.DatetimeProlepticGregorian(2023, 1, 1, 0, 0, 0, 0, has\PYGZus{}year\PYGZus{}zero=True)]],
      shape=(2022, 2), dtype=object)
Coordinates:
  * time     (time) object 16kB 0001\PYGZhy{}07\PYGZhy{}02 12:00:00 ... 2022\PYGZhy{}07\PYGZhy{}02 12:00:00
Dimensions without coordinates: bnds
\end{sphinxVerbatim}

\end{sphinxuseclass}\end{sphinxVerbatimOutput}

\end{sphinxuseclass}
\sphinxAtStartPar
As a result of the time average that the data represents,
it is inappropriate to plot this data
using a line plot
(the mean of the lines joining the points
is not the same as the data given in the files).
Instead, the data should be plotted (and used)
as a scatter or a step plot, as shown below.
(The same logic applies to any spatial plots
which could be created from our datasets
that include spatial dimensions).

\begin{sphinxuseclass}{cell}
\begin{sphinxuseclass}{tag_remove_input}\begin{sphinxVerbatimOutput}

\begin{sphinxuseclass}{cell_output}
\noindent\sphinxincludegraphics{{4b55fadb5777f8edd43f9e3fa4c3ab5264e585a4082389e1f50bfbf2161cb351}.png}

\end{sphinxuseclass}\end{sphinxVerbatimOutput}

\end{sphinxuseclass}
\end{sphinxuseclass}

\section{Monthly\sphinxhyphen{}, global\sphinxhyphen{}mean data}
\label{\detokenize{user-guide-historical:monthly-global-mean-data}}
\sphinxAtStartPar
If you want to have information at a finer level
of temporal detail, we also provide monthly files.
Like the global datasets, these come in three files.

\sphinxAtStartPar
For monthly data, the time labels in the filename are months.
Below we show the filenames for the CO2 output.

\begin{sphinxuseclass}{cell}
\begin{sphinxuseclass}{tag_remove_input}\begin{sphinxVerbatimOutput}

\begin{sphinxuseclass}{cell_output}
\begin{sphinxVerbatim}[commandchars=\\\{\}]
\PYGZhy{} co2\PYGZus{}input4MIPs\PYGZus{}GHGConcentrations\PYGZus{}CMIP\PYGZus{}CR\PYGZhy{}CMIP\PYGZhy{}1\PYGZhy{}0\PYGZhy{}0\PYGZus{}gm\PYGZus{}000101\PYGZhy{}099912.nc
\PYGZhy{} co2\PYGZus{}input4MIPs\PYGZus{}GHGConcentrations\PYGZus{}CMIP\PYGZus{}CR\PYGZhy{}CMIP\PYGZhy{}1\PYGZhy{}0\PYGZhy{}0\PYGZus{}gm\PYGZus{}100001\PYGZhy{}174912.nc
\PYGZhy{} co2\PYGZus{}input4MIPs\PYGZus{}GHGConcentrations\PYGZus{}CMIP\PYGZus{}CR\PYGZhy{}CMIP\PYGZhy{}1\PYGZhy{}0\PYGZhy{}0\PYGZus{}gm\PYGZus{}175001\PYGZhy{}202212.nc
\end{sphinxVerbatim}

\end{sphinxuseclass}\end{sphinxVerbatimOutput}

\end{sphinxuseclass}
\end{sphinxuseclass}
\sphinxAtStartPar
Again, the data can be trivially loaded with \sphinxhref{https://github.com/pydata/xarray}{xarray}.

\begin{sphinxuseclass}{cell}\begin{sphinxVerbatimInput}

\begin{sphinxuseclass}{cell_input}
\begin{sphinxVerbatim}[commandchars=\\\{\}]
\PYG{n}{ds\PYGZus{}co2\PYGZus{}monthly\PYGZus{}global} \PYG{o}{=} \PYG{n}{xr}\PYG{o}{.}\PYG{n}{open\PYGZus{}mfdataset}\PYG{p}{(}
    \PYG{n}{co2\PYGZus{}monthly\PYGZus{}global\PYGZus{}fps}\PYG{p}{,} \PYG{n}{decode\PYGZus{}times}\PYG{o}{=}\PYG{n}{time\PYGZus{}coder}
\PYG{p}{)}
\end{sphinxVerbatim}

\end{sphinxuseclass}\end{sphinxVerbatimInput}

\end{sphinxuseclass}
\begin{sphinxuseclass}{cell}\begin{sphinxVerbatimInput}

\begin{sphinxuseclass}{cell_input}
\begin{sphinxVerbatim}[commandchars=\\\{\}]
\PYG{n}{ds\PYGZus{}co2\PYGZus{}monthly\PYGZus{}global}
\end{sphinxVerbatim}

\end{sphinxuseclass}\end{sphinxVerbatimInput}
\begin{sphinxVerbatimOutput}

\begin{sphinxuseclass}{cell_output}
\begin{sphinxVerbatim}[commandchars=\\\{\}]
\PYGZlt{}xarray.Dataset\PYGZgt{} Size: 679kB
Dimensions:    (time: 24264, bnds: 2)
Coordinates:
  * time       (time) object 194kB 0001\PYGZhy{}01\PYGZhy{}15 00:00:00 ... 2022\PYGZhy{}12\PYGZhy{}15 00:00:00
Dimensions without coordinates: bnds
Data variables:
    co2        (time) float32 97kB 281.1 281.6 281.9 282.1 ... 416.7 418.2 419.3
    time\PYGZus{}bnds  (time, bnds) object 388kB 0001\PYGZhy{}01\PYGZhy{}01 00:00:00 ... 2023\PYGZhy{}01\PYGZhy{}01 0...
Attributes: (12/29)
    Conventions:             CF\PYGZhy{}1.7
    activity\PYGZus{}id:             input4MIPs
    comment:                 Data compiled by Climate Resource, based on scie...
    contact:                 zebedee.nicholls@climate\PYGZhy{}resource.com;malte.mein...
    creation\PYGZus{}date:           2025\PYGZhy{}02\PYGZhy{}28T18:23:29Z
    dataset\PYGZus{}category:        GHGConcentrations
    ...                      ...
    references\PYGZus{}urls:         https://scrippsco2.ucsd.edu/data/atmospheric\PYGZus{}co2...
    source\PYGZus{}id:               CR\PYGZhy{}CMIP\PYGZhy{}1\PYGZhy{}0\PYGZhy{}0
    source\PYGZus{}version:          1.0.0
    target\PYGZus{}mip:              CMIP
    tracking\PYGZus{}id:             hdl:21.14100/1b88c4f6\PYGZhy{}477d\PYGZhy{}46b3\PYGZhy{}8fcb\PYGZhy{}0f21170ea735
    variable\PYGZus{}id:             co2
\end{sphinxVerbatim}

\end{sphinxuseclass}\end{sphinxVerbatimOutput}

\end{sphinxuseclass}
\sphinxAtStartPar
For this data, the time bounds show that each point
is the average a month, not a year.

\begin{sphinxuseclass}{cell}\begin{sphinxVerbatimInput}

\begin{sphinxuseclass}{cell_input}
\begin{sphinxVerbatim}[commandchars=\\\{\}]
\PYG{n}{ds\PYGZus{}co2\PYGZus{}monthly\PYGZus{}global}\PYG{p}{[}\PYG{l+s+s2}{\PYGZdq{}}\PYG{l+s+s2}{time\PYGZus{}bnds}\PYG{l+s+s2}{\PYGZdq{}}\PYG{p}{]}
\end{sphinxVerbatim}

\end{sphinxuseclass}\end{sphinxVerbatimInput}
\begin{sphinxVerbatimOutput}

\begin{sphinxuseclass}{cell_output}
\begin{sphinxVerbatim}[commandchars=\\\{\}]
\PYGZlt{}xarray.DataArray \PYGZsq{}time\PYGZus{}bnds\PYGZsq{} (time: 24264, bnds: 2)\PYGZgt{} Size: 388kB
array([[cftime.DatetimeProlepticGregorian(1, 1, 1, 0, 0, 0, 0, has\PYGZus{}year\PYGZus{}zero=True),
        cftime.DatetimeProlepticGregorian(1, 2, 1, 0, 0, 0, 0, has\PYGZus{}year\PYGZus{}zero=True)],
       [cftime.DatetimeProlepticGregorian(1, 2, 1, 0, 0, 0, 0, has\PYGZus{}year\PYGZus{}zero=True),
        cftime.DatetimeProlepticGregorian(1, 3, 1, 0, 0, 0, 0, has\PYGZus{}year\PYGZus{}zero=True)],
       [cftime.DatetimeProlepticGregorian(1, 3, 1, 0, 0, 0, 0, has\PYGZus{}year\PYGZus{}zero=True),
        cftime.DatetimeProlepticGregorian(1, 4, 1, 0, 0, 0, 0, has\PYGZus{}year\PYGZus{}zero=True)],
       ...,
       [cftime.DatetimeProlepticGregorian(2022, 10, 1, 0, 0, 0, 0, has\PYGZus{}year\PYGZus{}zero=True),
        cftime.DatetimeProlepticGregorian(2022, 11, 1, 0, 0, 0, 0, has\PYGZus{}year\PYGZus{}zero=True)],
       [cftime.DatetimeProlepticGregorian(2022, 11, 1, 0, 0, 0, 0, has\PYGZus{}year\PYGZus{}zero=True),
        cftime.DatetimeProlepticGregorian(2022, 12, 1, 0, 0, 0, 0, has\PYGZus{}year\PYGZus{}zero=True)],
       [cftime.DatetimeProlepticGregorian(2022, 12, 1, 0, 0, 0, 0, has\PYGZus{}year\PYGZus{}zero=True),
        cftime.DatetimeProlepticGregorian(2023, 1, 1, 0, 0, 0, 0, has\PYGZus{}year\PYGZus{}zero=True)]],
      shape=(24264, 2), dtype=object)
Coordinates:
  * time     (time) object 194kB 0001\PYGZhy{}01\PYGZhy{}15 00:00:00 ... 2022\PYGZhy{}12\PYGZhy{}15 00:00:00
Dimensions without coordinates: bnds
\end{sphinxVerbatim}

\end{sphinxuseclass}\end{sphinxVerbatimOutput}

\end{sphinxuseclass}
\sphinxAtStartPar
As above, as a result of the time average that the data represents,
it is inappropriate to plot this data using a line plot.
Scatter or step plots should be used instead.

\begin{sphinxuseclass}{cell}
\begin{sphinxuseclass}{tag_remove_input}\begin{sphinxVerbatimOutput}

\begin{sphinxuseclass}{cell_output}
\noindent\sphinxincludegraphics{{62099885cabd725cc7209cc66a1a73347000aa729739ab0456d7f63b88412858}.png}

\end{sphinxuseclass}\end{sphinxVerbatimOutput}

\end{sphinxuseclass}
\end{sphinxuseclass}
\sphinxAtStartPar
The monthly data includes seasonality.
Plotting the monthly and yearly data
on the same axes makes particularly clear
why a line plot is inappropriate.

\begin{sphinxuseclass}{cell}
\begin{sphinxuseclass}{tag_remove_input}\begin{sphinxVerbatimOutput}

\begin{sphinxuseclass}{cell_output}
\noindent\sphinxincludegraphics{{2bffcce7cab8d19ff13233afaf61b1ab5910088aa53f801d99a4acb915b364c8}.png}

\end{sphinxuseclass}\end{sphinxVerbatimOutput}

\end{sphinxuseclass}
\end{sphinxuseclass}
\sphinxAtStartPar
At present, we do not provide data at a higher temporal resolution than monthly.
In theory, such a dataset is possible to compile,
however this requires careful consideration of daily
and potentially sub\sphinxhyphen{}daily trends (e.g. the diurnal cycle).


\section{Monthly\sphinxhyphen{}, latitudinally\sphinxhyphen{}resolved data}
\label{\detokenize{user-guide-historical:monthly-latitudinally-resolved-data}}
\sphinxAtStartPar
We also provide data with spatial,
specifically latituindal, resolution.
This data comes on a 15\sphinxhyphen{}degree latituindal grid
(see below for details of the grid and latitudinal bounds).
These files are identified by the grid label \sphinxcode{\sphinxupquote{gnz}}.
We only provide these files with monthly resolution.

\sphinxAtStartPar
For completeness, we note that we also provide hemispheric means.
These are not shown here,
but are identified by the grid label \sphinxcode{\sphinxupquote{gr1z}}.

\sphinxAtStartPar
Below we show the filenames for the latitudinally\sphinxhyphen{}resolved data
for CO2

\begin{sphinxuseclass}{cell}
\begin{sphinxuseclass}{tag_remove_input}\begin{sphinxVerbatimOutput}

\begin{sphinxuseclass}{cell_output}
\begin{sphinxVerbatim}[commandchars=\\\{\}]
\PYGZhy{} co2\PYGZus{}input4MIPs\PYGZus{}GHGConcentrations\PYGZus{}CMIP\PYGZus{}CR\PYGZhy{}CMIP\PYGZhy{}1\PYGZhy{}0\PYGZhy{}0\PYGZus{}gnz\PYGZus{}175001\PYGZhy{}202212.nc
\PYGZhy{} co2\PYGZus{}input4MIPs\PYGZus{}GHGConcentrations\PYGZus{}CMIP\PYGZus{}CR\PYGZhy{}CMIP\PYGZhy{}1\PYGZhy{}0\PYGZhy{}0\PYGZus{}gnz\PYGZus{}000101\PYGZhy{}099912.nc
\PYGZhy{} co2\PYGZus{}input4MIPs\PYGZus{}GHGConcentrations\PYGZus{}CMIP\PYGZus{}CR\PYGZhy{}CMIP\PYGZhy{}1\PYGZhy{}0\PYGZhy{}0\PYGZus{}gnz\PYGZus{}100001\PYGZhy{}174912.nc
\end{sphinxVerbatim}

\end{sphinxuseclass}\end{sphinxVerbatimOutput}

\end{sphinxuseclass}
\end{sphinxuseclass}
\sphinxAtStartPar
Again, the data can be trivially loaded with \sphinxhref{https://github.com/pydata/xarray}{xarray}.

\begin{sphinxuseclass}{cell}\begin{sphinxVerbatimInput}

\begin{sphinxuseclass}{cell_input}
\begin{sphinxVerbatim}[commandchars=\\\{\}]
\PYG{n}{ds\PYGZus{}co2\PYGZus{}monthly\PYGZus{}lat} \PYG{o}{=} \PYG{n}{xr}\PYG{o}{.}\PYG{n}{open\PYGZus{}mfdataset}\PYG{p}{(}
    \PYG{n}{co2\PYGZus{}monthly\PYGZus{}lat\PYGZus{}fps}\PYG{p}{,} \PYG{n}{decode\PYGZus{}times}\PYG{o}{=}\PYG{n}{time\PYGZus{}coder}\PYG{p}{,} \PYG{n}{data\PYGZus{}vars}\PYG{o}{=}\PYG{k+kc}{None}\PYG{p}{,} \PYG{n}{compat}\PYG{o}{=}\PYG{l+s+s2}{\PYGZdq{}}\PYG{l+s+s2}{no\PYGZus{}conflicts}\PYG{l+s+s2}{\PYGZdq{}}
\PYG{p}{)}
\end{sphinxVerbatim}

\end{sphinxuseclass}\end{sphinxVerbatimInput}

\end{sphinxuseclass}
\begin{sphinxuseclass}{cell}\begin{sphinxVerbatimInput}

\begin{sphinxuseclass}{cell_input}
\begin{sphinxVerbatim}[commandchars=\\\{\}]
\PYG{n}{ds\PYGZus{}co2\PYGZus{}monthly\PYGZus{}lat}
\end{sphinxVerbatim}

\end{sphinxuseclass}\end{sphinxVerbatimInput}
\begin{sphinxVerbatimOutput}

\begin{sphinxuseclass}{cell_output}
\begin{sphinxVerbatim}[commandchars=\\\{\}]
\PYGZlt{}xarray.Dataset\PYGZgt{} Size: 2MB
Dimensions:    (time: 24264, lat: 12, bnds: 2)
Coordinates:
  * time       (time) object 194kB 0001\PYGZhy{}01\PYGZhy{}15 00:00:00 ... 2022\PYGZhy{}12\PYGZhy{}15 00:00:00
  * lat        (lat) float64 96B \PYGZhy{}82.5 \PYGZhy{}67.5 \PYGZhy{}52.5 \PYGZhy{}37.5 ... 37.5 52.5 67.5 82.5
Dimensions without coordinates: bnds
Data variables:
    co2        (time, lat) float32 1MB 280.9 280.8 280.8 ... 425.3 424.6 424.1
    time\PYGZus{}bnds  (time, bnds) object 388kB 0001\PYGZhy{}01\PYGZhy{}01 00:00:00 ... 2023\PYGZhy{}01\PYGZhy{}01 0...
    lat\PYGZus{}bnds   (lat, bnds) float64 192B \PYGZhy{}90.0 \PYGZhy{}75.0 \PYGZhy{}75.0 ... 75.0 75.0 90.0
Attributes: (12/29)
    Conventions:             CF\PYGZhy{}1.7
    activity\PYGZus{}id:             input4MIPs
    comment:                 Data compiled by Climate Resource, based on scie...
    contact:                 zebedee.nicholls@climate\PYGZhy{}resource.com;malte.mein...
    creation\PYGZus{}date:           2025\PYGZhy{}02\PYGZhy{}28T18:23:23Z
    dataset\PYGZus{}category:        GHGConcentrations
    ...                      ...
    references\PYGZus{}urls:         https://scrippsco2.ucsd.edu/data/atmospheric\PYGZus{}co2...
    source\PYGZus{}id:               CR\PYGZhy{}CMIP\PYGZhy{}1\PYGZhy{}0\PYGZhy{}0
    source\PYGZus{}version:          1.0.0
    target\PYGZus{}mip:              CMIP
    tracking\PYGZus{}id:             hdl:21.14100/f6635404\PYGZhy{}8a1a\PYGZhy{}4aa9\PYGZhy{}918d\PYGZhy{}3792e8321f04
    variable\PYGZus{}id:             co2
\end{sphinxVerbatim}

\end{sphinxuseclass}\end{sphinxVerbatimOutput}

\end{sphinxuseclass}
\sphinxAtStartPar
For this data, the latitudinal bounds show the area
over which each point is the average.

\begin{sphinxuseclass}{cell}\begin{sphinxVerbatimInput}

\begin{sphinxuseclass}{cell_input}
\begin{sphinxVerbatim}[commandchars=\\\{\}]
\PYG{n}{ds\PYGZus{}co2\PYGZus{}monthly\PYGZus{}lat}\PYG{p}{[}\PYG{l+s+s2}{\PYGZdq{}}\PYG{l+s+s2}{lat\PYGZus{}bnds}\PYG{l+s+s2}{\PYGZdq{}}\PYG{p}{]}
\end{sphinxVerbatim}

\end{sphinxuseclass}\end{sphinxVerbatimInput}
\begin{sphinxVerbatimOutput}

\begin{sphinxuseclass}{cell_output}
\begin{sphinxVerbatim}[commandchars=\\\{\}]
\PYGZlt{}xarray.DataArray \PYGZsq{}lat\PYGZus{}bnds\PYGZsq{} (lat: 12, bnds: 2)\PYGZgt{} Size: 192B
array([[\PYGZhy{}90., \PYGZhy{}75.],
       [\PYGZhy{}75., \PYGZhy{}60.],
       [\PYGZhy{}60., \PYGZhy{}45.],
       [\PYGZhy{}45., \PYGZhy{}30.],
       [\PYGZhy{}30., \PYGZhy{}15.],
       [\PYGZhy{}15.,   0.],
       [  0.,  15.],
       [ 15.,  30.],
       [ 30.,  45.],
       [ 45.,  60.],
       [ 60.,  75.],
       [ 75.,  90.]])
Coordinates:
  * lat      (lat) float64 96B \PYGZhy{}82.5 \PYGZhy{}67.5 \PYGZhy{}52.5 \PYGZhy{}37.5 ... 37.5 52.5 67.5 82.5
Dimensions without coordinates: bnds
\end{sphinxVerbatim}

\end{sphinxuseclass}\end{sphinxVerbatimOutput}

\end{sphinxuseclass}
\sphinxAtStartPar
As above, but this time for the spatial axis,
it is inappropriate to plot this data using a line plot.
Scatter or step plots should be used instead.

\begin{sphinxuseclass}{cell}
\begin{sphinxuseclass}{tag_remove_input}\begin{sphinxVerbatimOutput}

\begin{sphinxuseclass}{cell_output}
\noindent\sphinxincludegraphics{{9b648bc4ec88caa61bdb85e2798999e633cf2bb00c8b26c36dd65c1b3c948af5}.png}

\end{sphinxuseclass}\end{sphinxVerbatimOutput}

\end{sphinxuseclass}
\end{sphinxuseclass}
\sphinxAtStartPar
We can compare the global\sphinxhyphen{}mean data
to the data at each latitude.
The strength of the latitudinal gradient varies also by gas (not shown).

\begin{sphinxuseclass}{cell}
\begin{sphinxuseclass}{tag_remove_input}\begin{sphinxVerbatimOutput}

\begin{sphinxuseclass}{cell_output}
\noindent\sphinxincludegraphics{{345f8d62539156b0c12da78e6060984f2b8917cd6d73901f48dc0f536349922d}.png}

\end{sphinxuseclass}\end{sphinxVerbatimOutput}

\end{sphinxuseclass}
\end{sphinxuseclass}
\sphinxAtStartPar
The data can also be plotted in a so\sphinxhyphen{}called “magic carpet”
to see the variation in space and time simultaneously.

\begin{sphinxuseclass}{cell}
\begin{sphinxuseclass}{tag_remove_input}\begin{sphinxVerbatimOutput}

\begin{sphinxuseclass}{cell_output}
\noindent\sphinxincludegraphics{{5b8f1d2d026f765dea634561bd65d1d5430c92c7e1947b7dc04e828554188b77}.png}

\end{sphinxuseclass}\end{sphinxVerbatimOutput}

\end{sphinxuseclass}
\end{sphinxuseclass}

\section{Differences from CMIP6}
\label{\detokenize{user-guide-historical:differences-from-cmip6}}

\subsection{File formats and naming}
\label{\detokenize{user-guide-historical:file-formats-and-naming}}
\sphinxAtStartPar
The file formats are generally close to CMIP6.
There are three key changes:
\begin{enumerate}
\sphinxsetlistlabels{\arabic}{enumi}{enumii}{}{.}%
\item {}
\sphinxAtStartPar
we have split the global\sphinxhyphen{}mean and hemispheric\sphinxhyphen{}mean data into separate files.
In CMIP6, this data was in the same file (with a grid label of \sphinxcode{\sphinxupquote{GMNHSH}}).
We have split this for two reasons:
a) \sphinxcode{\sphinxupquote{GMNHSH}} is not a grid label recognised in the CMIP CVs {[}REF\sphinxhyphen{}TODO{]} and
b) having global\sphinxhyphen{}mean and hemispheric\sphinxhyphen{}mean data in the same file
required us to introduce a ‘sector’ coordinate,
which was confusing and does not follow the CF\sphinxhyphen{}conventions.

\item {}
\sphinxAtStartPar
we have split the files into different time components.
One file goes from year 1 to year 999 (inclusive).
The next file goes from year 1000 to year 1749 (inclusive).
The last file goes from year 1750 to year 2022 (inclusive).
This simplifies handling and allows groups to avoid loading data
they are not interested in (for CMIP, this generally means data pre\sphinxhyphen{}1750).

\item {}
\sphinxAtStartPar
we have simplified the names of all the variables.
They are now simply the names of the gases,
for example we now use “co2” rather than “mole\_fraction\_of\_carbon\_dioxide”.
A full mapping is provided below.

\end{enumerate}

\sphinxAtStartPar
There is one more minor change.
The data now starts in year one, rather than year zero.
We do this because year zero doesn’t exist in most calendars
(and we want to avoid users of the data having to hack around this
when using standard data analysis tools).


\subsubsection{Variable name mapping}
\label{\detokenize{user-guide-historical:variable-name-mapping}}
\begin{sphinxVerbatim}[commandchars=\\\{\}]
\PYG{n}{CMIP6\PYGZus{}TO\PYGZus{}CMIP7\PYGZus{}VARIABLE\PYGZus{}MAP} \PYG{o}{=} \PYG{p}{\PYGZob{}}
    \PYG{c+c1}{\PYGZsh{} name in CMIP6: name in CMIP7}
    \PYG{l+s+s2}{\PYGZdq{}}\PYG{l+s+s2}{mole\PYGZus{}fraction\PYGZus{}of\PYGZus{}carbon\PYGZus{}dioxide\PYGZus{}in\PYGZus{}air}\PYG{l+s+s2}{\PYGZdq{}}\PYG{p}{:} \PYG{l+s+s2}{\PYGZdq{}}\PYG{l+s+s2}{co2}\PYG{l+s+s2}{\PYGZdq{}}\PYG{p}{,}
    \PYG{l+s+s2}{\PYGZdq{}}\PYG{l+s+s2}{mole\PYGZus{}fraction\PYGZus{}of\PYGZus{}methane\PYGZus{}in\PYGZus{}air}\PYG{l+s+s2}{\PYGZdq{}}\PYG{p}{:} \PYG{l+s+s2}{\PYGZdq{}}\PYG{l+s+s2}{ch4}\PYG{l+s+s2}{\PYGZdq{}}\PYG{p}{,}
    \PYG{l+s+s2}{\PYGZdq{}}\PYG{l+s+s2}{mole\PYGZus{}fraction\PYGZus{}of\PYGZus{}nitrous\PYGZus{}oxide\PYGZus{}in\PYGZus{}air}\PYG{l+s+s2}{\PYGZdq{}}\PYG{p}{:} \PYG{l+s+s2}{\PYGZdq{}}\PYG{l+s+s2}{n2o}\PYG{l+s+s2}{\PYGZdq{}}\PYG{p}{,}
    \PYG{l+s+s2}{\PYGZdq{}}\PYG{l+s+s2}{mole\PYGZus{}fraction\PYGZus{}of\PYGZus{}c2f6\PYGZus{}in\PYGZus{}air}\PYG{l+s+s2}{\PYGZdq{}}\PYG{p}{:} \PYG{l+s+s2}{\PYGZdq{}}\PYG{l+s+s2}{c2f6}\PYG{l+s+s2}{\PYGZdq{}}\PYG{p}{,}
    \PYG{l+s+s2}{\PYGZdq{}}\PYG{l+s+s2}{mole\PYGZus{}fraction\PYGZus{}of\PYGZus{}c3f8\PYGZus{}in\PYGZus{}air}\PYG{l+s+s2}{\PYGZdq{}}\PYG{p}{:} \PYG{l+s+s2}{\PYGZdq{}}\PYG{l+s+s2}{c3f8}\PYG{l+s+s2}{\PYGZdq{}}\PYG{p}{,}
    \PYG{l+s+s2}{\PYGZdq{}}\PYG{l+s+s2}{mole\PYGZus{}fraction\PYGZus{}of\PYGZus{}c4f10\PYGZus{}in\PYGZus{}air}\PYG{l+s+s2}{\PYGZdq{}}\PYG{p}{:} \PYG{l+s+s2}{\PYGZdq{}}\PYG{l+s+s2}{c4f10}\PYG{l+s+s2}{\PYGZdq{}}\PYG{p}{,}
    \PYG{l+s+s2}{\PYGZdq{}}\PYG{l+s+s2}{mole\PYGZus{}fraction\PYGZus{}of\PYGZus{}c5f12\PYGZus{}in\PYGZus{}air}\PYG{l+s+s2}{\PYGZdq{}}\PYG{p}{:} \PYG{l+s+s2}{\PYGZdq{}}\PYG{l+s+s2}{c5f12}\PYG{l+s+s2}{\PYGZdq{}}\PYG{p}{,}
    \PYG{l+s+s2}{\PYGZdq{}}\PYG{l+s+s2}{mole\PYGZus{}fraction\PYGZus{}of\PYGZus{}c6f14\PYGZus{}in\PYGZus{}air}\PYG{l+s+s2}{\PYGZdq{}}\PYG{p}{:} \PYG{l+s+s2}{\PYGZdq{}}\PYG{l+s+s2}{c6f14}\PYG{l+s+s2}{\PYGZdq{}}\PYG{p}{,}
    \PYG{l+s+s2}{\PYGZdq{}}\PYG{l+s+s2}{mole\PYGZus{}fraction\PYGZus{}of\PYGZus{}c7f16\PYGZus{}in\PYGZus{}air}\PYG{l+s+s2}{\PYGZdq{}}\PYG{p}{:} \PYG{l+s+s2}{\PYGZdq{}}\PYG{l+s+s2}{c7f16}\PYG{l+s+s2}{\PYGZdq{}}\PYG{p}{,}
    \PYG{l+s+s2}{\PYGZdq{}}\PYG{l+s+s2}{mole\PYGZus{}fraction\PYGZus{}of\PYGZus{}c8f18\PYGZus{}in\PYGZus{}air}\PYG{l+s+s2}{\PYGZdq{}}\PYG{p}{:} \PYG{l+s+s2}{\PYGZdq{}}\PYG{l+s+s2}{c8f18}\PYG{l+s+s2}{\PYGZdq{}}\PYG{p}{,}
    \PYG{l+s+s2}{\PYGZdq{}}\PYG{l+s+s2}{mole\PYGZus{}fraction\PYGZus{}of\PYGZus{}c\PYGZus{}c4f8\PYGZus{}in\PYGZus{}air}\PYG{l+s+s2}{\PYGZdq{}}\PYG{p}{:} \PYG{l+s+s2}{\PYGZdq{}}\PYG{l+s+s2}{cc4f8}\PYG{l+s+s2}{\PYGZdq{}}\PYG{p}{,}
    \PYG{l+s+s2}{\PYGZdq{}}\PYG{l+s+s2}{mole\PYGZus{}fraction\PYGZus{}of\PYGZus{}carbon\PYGZus{}tetrachloride\PYGZus{}in\PYGZus{}air}\PYG{l+s+s2}{\PYGZdq{}}\PYG{p}{:} \PYG{l+s+s2}{\PYGZdq{}}\PYG{l+s+s2}{ccl4}\PYG{l+s+s2}{\PYGZdq{}}\PYG{p}{,}
    \PYG{l+s+s2}{\PYGZdq{}}\PYG{l+s+s2}{mole\PYGZus{}fraction\PYGZus{}of\PYGZus{}cf4\PYGZus{}in\PYGZus{}air}\PYG{l+s+s2}{\PYGZdq{}}\PYG{p}{:} \PYG{l+s+s2}{\PYGZdq{}}\PYG{l+s+s2}{cf4}\PYG{l+s+s2}{\PYGZdq{}}\PYG{p}{,}
    \PYG{l+s+s2}{\PYGZdq{}}\PYG{l+s+s2}{mole\PYGZus{}fraction\PYGZus{}of\PYGZus{}cfc11\PYGZus{}in\PYGZus{}air}\PYG{l+s+s2}{\PYGZdq{}}\PYG{p}{:} \PYG{l+s+s2}{\PYGZdq{}}\PYG{l+s+s2}{cfc11}\PYG{l+s+s2}{\PYGZdq{}}\PYG{p}{,}
    \PYG{l+s+s2}{\PYGZdq{}}\PYG{l+s+s2}{mole\PYGZus{}fraction\PYGZus{}of\PYGZus{}cfc113\PYGZus{}in\PYGZus{}air}\PYG{l+s+s2}{\PYGZdq{}}\PYG{p}{:} \PYG{l+s+s2}{\PYGZdq{}}\PYG{l+s+s2}{cfc113}\PYG{l+s+s2}{\PYGZdq{}}\PYG{p}{,}
    \PYG{l+s+s2}{\PYGZdq{}}\PYG{l+s+s2}{mole\PYGZus{}fraction\PYGZus{}of\PYGZus{}cfc114\PYGZus{}in\PYGZus{}air}\PYG{l+s+s2}{\PYGZdq{}}\PYG{p}{:} \PYG{l+s+s2}{\PYGZdq{}}\PYG{l+s+s2}{cfc114}\PYG{l+s+s2}{\PYGZdq{}}\PYG{p}{,}
    \PYG{l+s+s2}{\PYGZdq{}}\PYG{l+s+s2}{mole\PYGZus{}fraction\PYGZus{}of\PYGZus{}cfc115\PYGZus{}in\PYGZus{}air}\PYG{l+s+s2}{\PYGZdq{}}\PYG{p}{:} \PYG{l+s+s2}{\PYGZdq{}}\PYG{l+s+s2}{cfc115}\PYG{l+s+s2}{\PYGZdq{}}\PYG{p}{,}
    \PYG{l+s+s2}{\PYGZdq{}}\PYG{l+s+s2}{mole\PYGZus{}fraction\PYGZus{}of\PYGZus{}cfc12\PYGZus{}in\PYGZus{}air}\PYG{l+s+s2}{\PYGZdq{}}\PYG{p}{:} \PYG{l+s+s2}{\PYGZdq{}}\PYG{l+s+s2}{cfc12}\PYG{l+s+s2}{\PYGZdq{}}\PYG{p}{,}
    \PYG{l+s+s2}{\PYGZdq{}}\PYG{l+s+s2}{mole\PYGZus{}fraction\PYGZus{}of\PYGZus{}ch2cl2\PYGZus{}in\PYGZus{}air}\PYG{l+s+s2}{\PYGZdq{}}\PYG{p}{:} \PYG{l+s+s2}{\PYGZdq{}}\PYG{l+s+s2}{ch2cl2}\PYG{l+s+s2}{\PYGZdq{}}\PYG{p}{,}
    \PYG{l+s+s2}{\PYGZdq{}}\PYG{l+s+s2}{mole\PYGZus{}fraction\PYGZus{}of\PYGZus{}methyl\PYGZus{}bromide\PYGZus{}in\PYGZus{}air}\PYG{l+s+s2}{\PYGZdq{}}\PYG{p}{:} \PYG{l+s+s2}{\PYGZdq{}}\PYG{l+s+s2}{ch3br}\PYG{l+s+s2}{\PYGZdq{}}\PYG{p}{,}
    \PYG{l+s+s2}{\PYGZdq{}}\PYG{l+s+s2}{mole\PYGZus{}fraction\PYGZus{}of\PYGZus{}ch3ccl3\PYGZus{}in\PYGZus{}air}\PYG{l+s+s2}{\PYGZdq{}}\PYG{p}{:} \PYG{l+s+s2}{\PYGZdq{}}\PYG{l+s+s2}{ch3ccl3}\PYG{l+s+s2}{\PYGZdq{}}\PYG{p}{,}
    \PYG{l+s+s2}{\PYGZdq{}}\PYG{l+s+s2}{mole\PYGZus{}fraction\PYGZus{}of\PYGZus{}methyl\PYGZus{}chloride\PYGZus{}in\PYGZus{}air}\PYG{l+s+s2}{\PYGZdq{}}\PYG{p}{:} \PYG{l+s+s2}{\PYGZdq{}}\PYG{l+s+s2}{ch3cl}\PYG{l+s+s2}{\PYGZdq{}}\PYG{p}{,}
    \PYG{l+s+s2}{\PYGZdq{}}\PYG{l+s+s2}{mole\PYGZus{}fraction\PYGZus{}of\PYGZus{}chcl3\PYGZus{}in\PYGZus{}air}\PYG{l+s+s2}{\PYGZdq{}}\PYG{p}{:} \PYG{l+s+s2}{\PYGZdq{}}\PYG{l+s+s2}{chcl3}\PYG{l+s+s2}{\PYGZdq{}}\PYG{p}{,}
    \PYG{l+s+s2}{\PYGZdq{}}\PYG{l+s+s2}{mole\PYGZus{}fraction\PYGZus{}of\PYGZus{}halon1211\PYGZus{}in\PYGZus{}air}\PYG{l+s+s2}{\PYGZdq{}}\PYG{p}{:} \PYG{l+s+s2}{\PYGZdq{}}\PYG{l+s+s2}{halon1211}\PYG{l+s+s2}{\PYGZdq{}}\PYG{p}{,}
    \PYG{l+s+s2}{\PYGZdq{}}\PYG{l+s+s2}{mole\PYGZus{}fraction\PYGZus{}of\PYGZus{}halon1301\PYGZus{}in\PYGZus{}air}\PYG{l+s+s2}{\PYGZdq{}}\PYG{p}{:} \PYG{l+s+s2}{\PYGZdq{}}\PYG{l+s+s2}{halon1301}\PYG{l+s+s2}{\PYGZdq{}}\PYG{p}{,}
    \PYG{l+s+s2}{\PYGZdq{}}\PYG{l+s+s2}{mole\PYGZus{}fraction\PYGZus{}of\PYGZus{}halon2402\PYGZus{}in\PYGZus{}air}\PYG{l+s+s2}{\PYGZdq{}}\PYG{p}{:} \PYG{l+s+s2}{\PYGZdq{}}\PYG{l+s+s2}{halon2402}\PYG{l+s+s2}{\PYGZdq{}}\PYG{p}{,}
    \PYG{l+s+s2}{\PYGZdq{}}\PYG{l+s+s2}{mole\PYGZus{}fraction\PYGZus{}of\PYGZus{}hcfc141b\PYGZus{}in\PYGZus{}air}\PYG{l+s+s2}{\PYGZdq{}}\PYG{p}{:} \PYG{l+s+s2}{\PYGZdq{}}\PYG{l+s+s2}{hcfc141b}\PYG{l+s+s2}{\PYGZdq{}}\PYG{p}{,}
    \PYG{l+s+s2}{\PYGZdq{}}\PYG{l+s+s2}{mole\PYGZus{}fraction\PYGZus{}of\PYGZus{}hcfc142b\PYGZus{}in\PYGZus{}air}\PYG{l+s+s2}{\PYGZdq{}}\PYG{p}{:} \PYG{l+s+s2}{\PYGZdq{}}\PYG{l+s+s2}{hcfc142b}\PYG{l+s+s2}{\PYGZdq{}}\PYG{p}{,}
    \PYG{l+s+s2}{\PYGZdq{}}\PYG{l+s+s2}{mole\PYGZus{}fraction\PYGZus{}of\PYGZus{}hcfc22\PYGZus{}in\PYGZus{}air}\PYG{l+s+s2}{\PYGZdq{}}\PYG{p}{:} \PYG{l+s+s2}{\PYGZdq{}}\PYG{l+s+s2}{hcfc22}\PYG{l+s+s2}{\PYGZdq{}}\PYG{p}{,}
    \PYG{l+s+s2}{\PYGZdq{}}\PYG{l+s+s2}{mole\PYGZus{}fraction\PYGZus{}of\PYGZus{}hfc125\PYGZus{}in\PYGZus{}air}\PYG{l+s+s2}{\PYGZdq{}}\PYG{p}{:} \PYG{l+s+s2}{\PYGZdq{}}\PYG{l+s+s2}{hfc125}\PYG{l+s+s2}{\PYGZdq{}}\PYG{p}{,}
    \PYG{l+s+s2}{\PYGZdq{}}\PYG{l+s+s2}{mole\PYGZus{}fraction\PYGZus{}of\PYGZus{}hfc134a\PYGZus{}in\PYGZus{}air}\PYG{l+s+s2}{\PYGZdq{}}\PYG{p}{:} \PYG{l+s+s2}{\PYGZdq{}}\PYG{l+s+s2}{hfc134a}\PYG{l+s+s2}{\PYGZdq{}}\PYG{p}{,}
    \PYG{l+s+s2}{\PYGZdq{}}\PYG{l+s+s2}{mole\PYGZus{}fraction\PYGZus{}of\PYGZus{}hfc143a\PYGZus{}in\PYGZus{}air}\PYG{l+s+s2}{\PYGZdq{}}\PYG{p}{:} \PYG{l+s+s2}{\PYGZdq{}}\PYG{l+s+s2}{hfc143a}\PYG{l+s+s2}{\PYGZdq{}}\PYG{p}{,}
    \PYG{l+s+s2}{\PYGZdq{}}\PYG{l+s+s2}{mole\PYGZus{}fraction\PYGZus{}of\PYGZus{}hfc152a\PYGZus{}in\PYGZus{}air}\PYG{l+s+s2}{\PYGZdq{}}\PYG{p}{:} \PYG{l+s+s2}{\PYGZdq{}}\PYG{l+s+s2}{hfc152a}\PYG{l+s+s2}{\PYGZdq{}}\PYG{p}{,}
    \PYG{l+s+s2}{\PYGZdq{}}\PYG{l+s+s2}{mole\PYGZus{}fraction\PYGZus{}of\PYGZus{}hfc227ea\PYGZus{}in\PYGZus{}air}\PYG{l+s+s2}{\PYGZdq{}}\PYG{p}{:} \PYG{l+s+s2}{\PYGZdq{}}\PYG{l+s+s2}{hfc227ea}\PYG{l+s+s2}{\PYGZdq{}}\PYG{p}{,}
    \PYG{l+s+s2}{\PYGZdq{}}\PYG{l+s+s2}{mole\PYGZus{}fraction\PYGZus{}of\PYGZus{}hfc23\PYGZus{}in\PYGZus{}air}\PYG{l+s+s2}{\PYGZdq{}}\PYG{p}{:} \PYG{l+s+s2}{\PYGZdq{}}\PYG{l+s+s2}{hfc23}\PYG{l+s+s2}{\PYGZdq{}}\PYG{p}{,}
    \PYG{l+s+s2}{\PYGZdq{}}\PYG{l+s+s2}{mole\PYGZus{}fraction\PYGZus{}of\PYGZus{}hfc236fa\PYGZus{}in\PYGZus{}air}\PYG{l+s+s2}{\PYGZdq{}}\PYG{p}{:} \PYG{l+s+s2}{\PYGZdq{}}\PYG{l+s+s2}{hfc236fa}\PYG{l+s+s2}{\PYGZdq{}}\PYG{p}{,}
    \PYG{l+s+s2}{\PYGZdq{}}\PYG{l+s+s2}{mole\PYGZus{}fraction\PYGZus{}of\PYGZus{}hfc245fa\PYGZus{}in\PYGZus{}air}\PYG{l+s+s2}{\PYGZdq{}}\PYG{p}{:} \PYG{l+s+s2}{\PYGZdq{}}\PYG{l+s+s2}{hfc245fa}\PYG{l+s+s2}{\PYGZdq{}}\PYG{p}{,}
    \PYG{l+s+s2}{\PYGZdq{}}\PYG{l+s+s2}{mole\PYGZus{}fraction\PYGZus{}of\PYGZus{}hfc32\PYGZus{}in\PYGZus{}air}\PYG{l+s+s2}{\PYGZdq{}}\PYG{p}{:} \PYG{l+s+s2}{\PYGZdq{}}\PYG{l+s+s2}{hfc32}\PYG{l+s+s2}{\PYGZdq{}}\PYG{p}{,}
    \PYG{l+s+s2}{\PYGZdq{}}\PYG{l+s+s2}{mole\PYGZus{}fraction\PYGZus{}of\PYGZus{}hfc365mfc\PYGZus{}in\PYGZus{}air}\PYG{l+s+s2}{\PYGZdq{}}\PYG{p}{:} \PYG{l+s+s2}{\PYGZdq{}}\PYG{l+s+s2}{hfc365mfc}\PYG{l+s+s2}{\PYGZdq{}}\PYG{p}{,}
    \PYG{l+s+s2}{\PYGZdq{}}\PYG{l+s+s2}{mole\PYGZus{}fraction\PYGZus{}of\PYGZus{}hfc4310mee\PYGZus{}in\PYGZus{}air}\PYG{l+s+s2}{\PYGZdq{}}\PYG{p}{:} \PYG{l+s+s2}{\PYGZdq{}}\PYG{l+s+s2}{hfc4310mee}\PYG{l+s+s2}{\PYGZdq{}}\PYG{p}{,}
    \PYG{l+s+s2}{\PYGZdq{}}\PYG{l+s+s2}{mole\PYGZus{}fraction\PYGZus{}of\PYGZus{}nf3\PYGZus{}in\PYGZus{}air}\PYG{l+s+s2}{\PYGZdq{}}\PYG{p}{:} \PYG{l+s+s2}{\PYGZdq{}}\PYG{l+s+s2}{nf3}\PYG{l+s+s2}{\PYGZdq{}}\PYG{p}{,}
    \PYG{l+s+s2}{\PYGZdq{}}\PYG{l+s+s2}{mole\PYGZus{}fraction\PYGZus{}of\PYGZus{}sf6\PYGZus{}in\PYGZus{}air}\PYG{l+s+s2}{\PYGZdq{}}\PYG{p}{:} \PYG{l+s+s2}{\PYGZdq{}}\PYG{l+s+s2}{sf6}\PYG{l+s+s2}{\PYGZdq{}}\PYG{p}{,}
    \PYG{l+s+s2}{\PYGZdq{}}\PYG{l+s+s2}{mole\PYGZus{}fraction\PYGZus{}of\PYGZus{}so2f2\PYGZus{}in\PYGZus{}air}\PYG{l+s+s2}{\PYGZdq{}}\PYG{p}{:} \PYG{l+s+s2}{\PYGZdq{}}\PYG{l+s+s2}{so2f2}\PYG{l+s+s2}{\PYGZdq{}}\PYG{p}{,}
    \PYG{l+s+s2}{\PYGZdq{}}\PYG{l+s+s2}{mole\PYGZus{}fraction\PYGZus{}of\PYGZus{}cfc11eq\PYGZus{}in\PYGZus{}air}\PYG{l+s+s2}{\PYGZdq{}}\PYG{p}{:} \PYG{l+s+s2}{\PYGZdq{}}\PYG{l+s+s2}{cfc11eq}\PYG{l+s+s2}{\PYGZdq{}}\PYG{p}{,}
    \PYG{l+s+s2}{\PYGZdq{}}\PYG{l+s+s2}{mole\PYGZus{}fraction\PYGZus{}of\PYGZus{}cfc12eq\PYGZus{}in\PYGZus{}air}\PYG{l+s+s2}{\PYGZdq{}}\PYG{p}{:} \PYG{l+s+s2}{\PYGZdq{}}\PYG{l+s+s2}{cfc12eq}\PYG{l+s+s2}{\PYGZdq{}}\PYG{p}{,}
    \PYG{l+s+s2}{\PYGZdq{}}\PYG{l+s+s2}{mole\PYGZus{}fraction\PYGZus{}of\PYGZus{}hfc134aeq\PYGZus{}in\PYGZus{}air}\PYG{l+s+s2}{\PYGZdq{}}\PYG{p}{:} \PYG{l+s+s2}{\PYGZdq{}}\PYG{l+s+s2}{hfc134aeq}\PYG{l+s+s2}{\PYGZdq{}}\PYG{p}{,}
\PYG{p}{\PYGZcb{}}
\end{sphinxVerbatim}


\subsection{Data comparisons}
\label{\detokenize{user-guide-historical:data-comparisons}}
\sphinxAtStartPar
Comparing the data from CMIP6 and CMIP7 shows minor changes
(although doing this comparison requires a bit of care
because of the changes in file formats).


\subsubsection{Atmospheric concentrations: Year 1 \sphinxhyphen{} 2022}
\label{\detokenize{user-guide-historical:atmospheric-concentrations-year-1-2022}}
\begin{sphinxuseclass}{cell}
\begin{sphinxuseclass}{tag_remove_input}\begin{sphinxVerbatimOutput}

\begin{sphinxuseclass}{cell_output}
\noindent\sphinxincludegraphics{{4bc3fb671184c1c40ce9b678855996b1c9af187a7fae44b02e9c3e2fe4527939}.png}

\end{sphinxuseclass}\end{sphinxVerbatimOutput}

\end{sphinxuseclass}
\end{sphinxuseclass}

\subsubsection{Atmospheric concentrations: Year 1750 \sphinxhyphen{} 2022}
\label{\detokenize{user-guide-historical:atmospheric-concentrations-year-1750-2022}}
\begin{sphinxuseclass}{cell}
\begin{sphinxuseclass}{tag_remove_input}\begin{sphinxVerbatimOutput}

\begin{sphinxuseclass}{cell_output}
\noindent\sphinxincludegraphics{{c734582d75f4ff3a3c2b26da4079254abeee8c9f4a0ade96f6c1e7e82a7c3ac9}.png}

\end{sphinxuseclass}\end{sphinxVerbatimOutput}

\end{sphinxuseclass}
\end{sphinxuseclass}

\subsubsection{Atmospheric concentrations: Year 1957 \sphinxhyphen{} 2022}
\label{\detokenize{user-guide-historical:atmospheric-concentrations-year-1957-2022}}
\sphinxAtStartPar
1957 is the start of the Scripps ground\sphinxhyphen{}based record.
Before this, data is based on ice cores alone.

\begin{sphinxuseclass}{cell}
\begin{sphinxuseclass}{tag_remove_input}\begin{sphinxVerbatimOutput}

\begin{sphinxuseclass}{cell_output}
\noindent\sphinxincludegraphics{{99eedebd518653231026b96c57a2ee0d6d2a8f4fff324b882898ea2bb39416c6}.png}

\end{sphinxuseclass}\end{sphinxVerbatimOutput}

\end{sphinxuseclass}
\end{sphinxuseclass}

\subsubsection{Approximate radiative effect: Year 1 \sphinxhyphen{} 2022}
\label{\detokenize{user-guide-historical:approximate-radiative-effect-year-1-2022}}
\sphinxAtStartPar
As seen above, in atmospheric concentration terms
the differences are small.
However, this can be put on a common scale
by comparing the differences in radiative effect terms.
This gives an approximation of the size of the difference
that would be seen by an Earth System Model’s (ESM’s) radiation code.
This uses basic linear approximations,
assuming that the radiative effect of each gas
is simply its atmospheric concentration multiplied by a constant.
This isn’t the same as effective radiative forcing (ERF).
For that comparison, see the later sections focussed on ERF.

\sphinxAtStartPar
Values below come from Table 7.SM.7 of
IPCC AR7 WG1 Ch. 7 Supplementary Material%
\begin{footnote}[4]\sphinxAtStartFootnote
\sphinxurl{https://www.ipcc.ch/report/ar6/wg1/downloads/report/IPCC\_AR6\_WGI\_Chapter07\_SM.pdf}
%
\end{footnote}.

\begin{sphinxuseclass}{cell}\begin{sphinxVerbatimInput}

\begin{sphinxuseclass}{cell_input}
\begin{sphinxVerbatim}[commandchars=\\\{\}]
\PYG{k+kn}{from}\PYG{+w}{ }\PYG{n+nn}{openscm\PYGZus{}units}\PYG{+w}{ }\PYG{k+kn}{import} \PYG{n}{unit\PYGZus{}registry}

\PYG{n}{Q} \PYG{o}{=} \PYG{n}{unit\PYGZus{}registry}\PYG{o}{.}\PYG{n}{Quantity}

\PYG{n}{RADIATIVE\PYGZus{}EFFICIENCIES} \PYG{o}{=} \PYG{p}{\PYGZob{}}
    \PYG{l+s+s2}{\PYGZdq{}}\PYG{l+s+s2}{co2}\PYG{l+s+s2}{\PYGZdq{}}\PYG{p}{:} \PYG{n}{Q}\PYG{p}{(}\PYG{l+m+mf}{1.33e\PYGZhy{}5}\PYG{p}{,} \PYG{l+s+s2}{\PYGZdq{}}\PYG{l+s+s2}{W / m\PYGZca{}2 / ppb}\PYG{l+s+s2}{\PYGZdq{}}\PYG{p}{)}\PYG{p}{,}
    \PYG{l+s+s2}{\PYGZdq{}}\PYG{l+s+s2}{ch4}\PYG{l+s+s2}{\PYGZdq{}}\PYG{p}{:} \PYG{n}{Q}\PYG{p}{(}\PYG{l+m+mf}{3.88e\PYGZhy{}4}\PYG{p}{,} \PYG{l+s+s2}{\PYGZdq{}}\PYG{l+s+s2}{W / m\PYGZca{}2 / ppb}\PYG{l+s+s2}{\PYGZdq{}}\PYG{p}{)}\PYG{p}{,}
    \PYG{l+s+s2}{\PYGZdq{}}\PYG{l+s+s2}{n2o}\PYG{l+s+s2}{\PYGZdq{}}\PYG{p}{:} \PYG{n}{Q}\PYG{p}{(}\PYG{l+m+mf}{3.2e\PYGZhy{}3}\PYG{p}{,} \PYG{l+s+s2}{\PYGZdq{}}\PYG{l+s+s2}{W / m\PYGZca{}2 / ppb}\PYG{l+s+s2}{\PYGZdq{}}\PYG{p}{)}\PYG{p}{,}
    \PYG{l+s+s2}{\PYGZdq{}}\PYG{l+s+s2}{cfc12eq}\PYG{l+s+s2}{\PYGZdq{}}\PYG{p}{:} \PYG{n}{Q}\PYG{p}{(}\PYG{l+m+mf}{0.358}\PYG{p}{,} \PYG{l+s+s2}{\PYGZdq{}}\PYG{l+s+s2}{W / m\PYGZca{}2 / ppb}\PYG{l+s+s2}{\PYGZdq{}}\PYG{p}{)}\PYG{p}{,}
    \PYG{l+s+s2}{\PYGZdq{}}\PYG{l+s+s2}{hfc134aeq}\PYG{l+s+s2}{\PYGZdq{}}\PYG{p}{:} \PYG{n}{Q}\PYG{p}{(}\PYG{l+m+mf}{0.167}\PYG{p}{,} \PYG{l+s+s2}{\PYGZdq{}}\PYG{l+s+s2}{W / m\PYGZca{}2 / ppb}\PYG{l+s+s2}{\PYGZdq{}}\PYG{p}{)}\PYG{p}{,}
\PYG{p}{\PYGZcb{}}
\end{sphinxVerbatim}

\end{sphinxuseclass}\end{sphinxVerbatimInput}

\end{sphinxuseclass}
\begin{sphinxuseclass}{cell}
\begin{sphinxuseclass}{tag_remove_input}\begin{sphinxVerbatimOutput}

\begin{sphinxuseclass}{cell_output}
\noindent\sphinxincludegraphics{{f57ba5307fdcfd511ab16292c618b7c1ecccba25bb6021b3449cf9caeb5178b8}.png}

\end{sphinxuseclass}\end{sphinxVerbatimOutput}

\end{sphinxuseclass}
\end{sphinxuseclass}

\subsubsection{Approximate radiative effect: Year 1750 \sphinxhyphen{} 2022}
\label{\detokenize{user-guide-historical:approximate-radiative-effect-year-1750-2022}}
\sphinxAtStartPar
This is the period relevant for historical simulations in CMIP.

\begin{sphinxuseclass}{cell}
\begin{sphinxuseclass}{tag_remove_input}\begin{sphinxVerbatimOutput}

\begin{sphinxuseclass}{cell_output}
\noindent\sphinxincludegraphics{{fc733fc4bc281f801c29c6b9e042d90947d7d69b05ada9bb5011ba016ff16d14}.png}

\end{sphinxuseclass}\end{sphinxVerbatimOutput}

\end{sphinxuseclass}
\end{sphinxuseclass}

\subsubsection{Approximate effective radiative forcing: Year 1750 \sphinxhyphen{} 2022}
\label{\detokenize{user-guide-historical:approximate-effective-radiative-forcing-year-1750-2022}}
\sphinxAtStartPar
The above isn’t effective radiative forcing.
For that, you have to normalise the data to some reference year.
There are a few different choices for this reference year.
In IPCC reports, it is 1750 so that is what we show here.
It should be noted that some ESMs may make other choices,
but these would not have a great effect on the interpretation
of the difference between the CMIP6 and CMIP7 datasets.

\sphinxAtStartPar
Note that this approximation is linear,
which is a particularly strong approximation for CO2
because of its logarithmic forcing nature.
We show this approximation here nonetheless
because it provides an order of magnitude estimate
for the change from CMIP6 in ERF terms.
The forthcoming manuscripts will explore the subtleties
of this quantification in more detail.

\begin{sphinxuseclass}{cell}
\begin{sphinxuseclass}{tag_remove_input}\begin{sphinxVerbatimOutput}

\begin{sphinxuseclass}{cell_output}
\noindent\sphinxincludegraphics{{0e71048669369498d3ac7714e0de2eaf041ada254bb0a937422b6639e7c340c0}.png}

\end{sphinxuseclass}\end{sphinxVerbatimOutput}

\end{sphinxuseclass}
\end{sphinxuseclass}
\sphinxAtStartPar
In summary, in ERF terms, the differences from CMIP6 are very small.
For all gases, they are less than around 0.025 W / m2.
Compared to the estimated total greenhouse gas forcing and uncertainty in IPCC AR6
(see Section 7.3.5.2 of AR6 WG1 Chapter 7%
\begin{footnote}[5]\sphinxAtStartFootnote
\sphinxurl{https://www.ipcc.ch/report/ar6/wg1/chapter/chapter-7/}
%
\end{footnote}),
estimated to be 3.84 W / m2
(very likely range of 3.46 to 4.22 W / m2),
such differences are particularly small.


\subsubsection{Atmospheric concentrations including seasonality: Year 2000 \sphinxhyphen{} 2022}
\label{\detokenize{user-guide-historical:atmospheric-concentrations-including-seasonality-year-2000-2022}}
\sphinxAtStartPar
The final comparisons we show are atmospheric concentrations including seasonality.
Given that most greenhouse gases
are well\sphinxhyphen{}mixed with lifetimes much greater than a year,
these differences are unlikely to be of huge interest to ESMs.
However, for other applications, such seasonality differences may matter more.

\begin{sphinxuseclass}{cell}
\begin{sphinxuseclass}{tag_remove_input}\begin{sphinxVerbatimOutput}

\begin{sphinxuseclass}{cell_output}
\noindent\sphinxincludegraphics{{1397dbeb504eca5933838be2cdb5c123bf88cdfa076c26ae83cb6adda7af218a}.png}

\end{sphinxuseclass}\end{sphinxVerbatimOutput}

\end{sphinxuseclass}
\end{sphinxuseclass}
\sphinxAtStartPar
Like the annual\sphinxhyphen{}means,
the atmospheric concentrations including seasonality
are reasonably consistent between CMIP6 and CMIP7.
There are some areas of change.
Full details of these changes will be provided
in the forthcoming manuscripts.

\phantomsection\label{\detokenize{user-guide-historical:id6}}

\bigskip\hrule\bigskip








\renewcommand{\indexname}{Index}
\printindex
\printbibliography[heading=bibintoc,title={Whole bibliography}]
\end{document}
